\chapter{\ifproject%
\ifenglish Project Structure and Methodology\else โครงสร้างและขั้นตอนการทำงาน\fi
\else%
\ifenglish Project Structure\else โครงสร้างของโครงงาน\fi
\fi
}

\makeatletter

% \renewcommand\section{\@startsection {section}{1}{\z@}%
%                                    {13.5ex \@plus -1ex \@minus -.2ex}%
%                                    {2.3ex \@plus.2ex}%
%                                    {\normalfont\large\bfseries}}

\makeatother
%\vspace{2ex}
% \titleformat{\section}{\normalfont\bfseries}{\thesection}{1em}{}
% \titlespacing*{\section}{0pt}{10ex}{0pt}

\section{สถาปัตยกรรมระบบ}
  \qquad โครงงานนี้ได้ออกแบบสถาปัตยกรรมของระบบเป็นแบบ \textbf{Microservices} โดยบทนี้จะกล่าวถึง
    โครงสร้างของระบบทั้งหมด และอธิบายถึงแต่ละส่วนของระบบ โดยระบบจะถูกแบ่งออกเป็นส่วนย่อยๆ 
    แต่ละส่วนจะมีหน้าที่และความรับผิดชอบในการทำงานที่แตกต่างกัน

    \begin{figure}[H]
      \centering
      \includegraphics[width=0.8\textwidth]{image/Approach/Architecture.png}
      \caption[Architecture]{System Architecture}
      \label{fig:architecture}
    \end{figure}
    \FloatBarrier

  จากรูปที่ 3.1 แสดงถึงสถาปัตยกรรมของระบบที่ถูกแบ่งออกเป็นส่วนย่อยๆ ดังนี้
  \begin{enumerate}
    \item \textbf{Frontend} ส่วนนี้เป็นส่วนที่ทำหน้าที่ในการแสดงผลของระบบ โดยส่วนนี้จะถูกพัฒนาด้วย \textbf{Next.js}
    \item \textbf{Reverse Proxy} ส่วนนี้เป็นส่วนที่ทำหน้าที่ในการจัดการการเชื่อมต่อระหว่าง Frontend และ Backend
    โดยส่วนนั้นจะถูกพัฒนาด้วย \textbf{Nginx Proxy Manager(NPM)}
    \item \textbf{Backend} ส่วนนี้เป็นส่วนที่ทำหน้าที่ในการประมวลผลข้อมูลและให้บริการ API โดยส่วนนั้นจะถูกพัฒนาด้วย
    \textbf{Go (Golang)} และเฟรมเวิร์ก \textbf{Fiber}
    \item \textbf{OCR Service} ส่วนนี้เป็นส่วนที่ทำหน้าที่ในการประมวลผลภาพและทำการรู้จำตัวอักษร โดยส่วนนั้นจะถูกพัฒนาด้วย \textbf{Tesseract OCR}
    \item \textbf{Database and Storage} ส่วนนี้เป็นส่วนที่ทำหน้าที่ในการจัดเก็บข้อมูล โดยจะใช้ \textbf{PostgreSQL}
    เป็นฐานข้อมูลหลัก และใช้ \textbf{MinIO} สำหรับการจัดเก็บไฟล์
  \end{enumerate}

\section{การพิสูจน์ตัวตน (Authentication Services)}
  \qquad ในแพลตฟอร์มนี้ได้มีการใช้บริการการยืนยันตัวตนของผู้ใช้งาน (Authentication) เป็นส่วนสําคัญช่วยในเรื่องการให้
  ผู้ใช้แต่ละคนสามารถเข้าถึงข้อมูลและฟังก์ชันที่เหมาะสมกับบทบาทของตน ระบบนี้ใช้ Microsoft Entra ID
  (เดิมคือ Azure Active Directory หรือ Azure AD) เป็นโซลูชันหลักสำหรับการจัดการการตรวจสอบสิทธิ์ของผู้ใช้ โดยผู้ใช้
  สามารถเข้าสู่ระบบผ่าน CMU Account ซึ่งเป็นบัญชีที่ใช้ภายในมหาวิทยาลัยเชียงใหม่ และได้ใช้ Google OAuth 2.0
  เป็นอีกหนึ่งวิธีในการเข้าสู่ระบบผ่านบัญชี Google ของผู้ใช้ ซึ่งช่วยเพิ่มความสะดวกสบายในการเข้าถึงแพลตฟอร์มนี้
  \subsection{การทํางานของ Authentication}
    \qquad เมื่อผู้ใช้พยายามเข้าสู่ระบบผ่านทางแพลตฟอร์ม ระบบจะทําการตรวจสอบข้อมูลการเข้าสู่ระบบของผู้ใช้
    ผ่าน Microsoft Entra ID หรือ Google OAuth 2.0 ตามวิธีที่ผู้ใช้เลือก หลังจากที่ผู้ใช้ได้ทําการยืนยันตัวตน
    เรียบร้อยแล้ว ระบบจะทําการตรวจสอบสิทธิ์โดยใช้ข้อมูลจาก Microsoft Entra ID หรือ Google OAuth 2.0
    ดังต่อไปนี้
    \begin{enumerate}
      \item \textbf{ผู้ใช้เข้าถึงหน้าเข้าสู่ระบบ (Login Page)} เมื่อผู้ใช้ต้องการเข้าสู่แพลตฟอร์มสำหรับการตรวจใบงานแบบกระดาษ
      (PaperGrader) ระบบจะเปลี่ยนเส้นทาง (Redirect) ไปยังส่วนของหน้าล็อกอินของ Microsoft Entra ID เพื่อให้ผู้ใช้ป้อนข้อมูลรับรอง
      (CMU Account) หรือไปยังหน้าล็อกอินของ Google OAuth 2.0 เพื่อให้ผู้ใช้ป้อนข้อมูลบัญชี Google
      \item \textbf{ตรวจสอบสิทธิ์ด้วย OAuth 2.0 และ OpenID Connect} โดยที่ Microsoft Entra ID ใช้ OAuth 2.0 และ
      OpenID Connect ในขั้นตอนการตรวจสอบตัวตนของผู้ใช้ หลังจากที่ผู้ใช้กรอกข้อมูลรับรอง (Email และ Password) ระบบจะตรวจสอบความถูกต้องของบัญชีและสิทธิ์การเข้าถึง
      และในส่วนของ Google OAuth 2.0 ก็ใช้กระบวนการที่คล้ายกันโดยผู้ใช้จะต้องอนุญาตให้แอปพลิเคชันเข้าถึงข้อมูลบัญชี Google ของตน
      \item \textbf{ออกโทเค็นยืนยันตัวตน (Access Token \& JWT Token)} หากการตรวจสอบสําเร็จ Microsoft Entra ID
      จะออก Access Token ให้กับผู้ใช้ ซึ่งระบบจะใช้เพื่อตรวจสอบสิทธิ์และดึงข้อมูลบัญชีจาก Entra ID จากนั้น Backend จะใช้
      ข้อมูลที่ได้รับสร้าง JWT Token เพื่อใช้ในการตรวจสอบสิทธิ์ในแพลตฟอร์ม โดยที่
        \begin{itemize}
          \item \textbf{Access Token} เป็นโทเค็นที่ใช้ในการเข้าถึงทรัพยากรที่ได้รับอนุญาตจาก Microsoft Entra ID
          \item \textbf{JWT Token} เป็นโทเค็นที่ใช้ในการตรวจสอบสิทธิ์ภายในแพลตฟอร์ม โดยจะถูกแนบไปกับทุกคําขอของแพลตฟอร์ม 
          เพื่อตรวจสอบความถูกต้องของ Token และให้สิทธิ์เข้าถึงตามบทบาทของผู้ใช้
        \end{itemize}
      \item \textbf{ระบบตรวจสอบบทบาทของผู้ใช้ (Role-Based Access Control-RBAC)} เมื่อได้รับโทเค็นแล้วแพลตฟอร์ม 
      จะทําการตรวจสอบบทบาทของผู้ใช้ เช่น อาจารย์หรือนักศึกษา ผ่านค่าที่อยู่ในโทเค็น เพื่อกําหนดสิทธิ์การเข้าถึงข้อมูลและฟังก์ชันต่างๆ 
      ในแพลตฟอร์ม
      \item \textbf{ให้สิทธิ์การเข้าถึงแพลตฟอร์ม (Access Rights)} หากการตรวจสอบสิทธิ์สําเร็จ ผู้ใช้จะสามารถเข้าถึง
      แพลตฟอร์มตรวจใบงานแบบกระดาษ (PaperGrader) ได้ตามบทบาทที่กําหนดไว้ โดยผู้ใช้จะสามารถใช้งานฟีเจอร์ต่างๆ
      ได้ตามบทบาทที่ได้รับ เช่น อาจารย์สามารถอัปโหลดไฟล์ใบงาน นักศึกษาสามารถดูผลคะแนนของใบงาน
    \end{enumerate}

\section{โครงสร้างฐานข้อมูล (Database Schema)}
  \begin{figure}[H]
    \centering
    \includegraphics[width=1\textwidth]{image/Approach/Database-Schema.png}
    \caption[Database Schema]{Database Schema}
    \label{fig:database_schema}
  \end{figure}
  \FloatBarrier
  \qquad ฐานข้อมูลที่ใช้ในโครงงานนี้ คือ PostgreSQL โดยมีการออกแบบโครงสร้างฐานข้อมูล (Database Schema) เพื่อรองรับการทำงานของแพลตฟอร์มอย่างมีประสิทธิภาพ
  โดยโครงสร้างฐานข้อมูลประกอบด้วยตารางหลักๆ ดังนี้
  \begin{enumerate}
    \item \textbf{Universities} ตารางสำหรับเก็บข้อมูลรายการมหาวิทยาลัย
    \item \textbf{User Groups} ตารางสำหรับเก็บข้อมูลประเภทกลุ่มของผู้ใช้งานในแพลตฟอร์ม
    \item \textbf{Users} ตารางสำหรับเก็บข้อมูลต่างๆ ของผู้ใช้งานในแพลตฟอร์ม
    \item \textbf{Courses} ตารางสำหรับเก็บข้อมูลกระบวนวิชา
    \item \textbf{Sections} ตารางสำหรับเก็บข้อมูลลำดับตอนภายในกระบวนวิชา
    \item \textbf{Enrollment lists} ตารางสำหรับเก็บข้อมูลการลงทะเบียนของผู้ใช้งานในกระบวนวิชา
    \item \textbf{Personal Data} ตารางสำหรับเก็บข้อมูลของผู้ใช้งานในกระบวนวิชาที่ได้ลงทะเบียน
    \item \textbf{Assignments} ตารางสำหรับเก็บข้อมูลการมอบหมายงาน
    \item \textbf{Assignment Files} ตารางสำหรับเก็บข้อมูลไฟล์การมอบหมายงาน
    \item \textbf{Assignment Sections} ตารางสำหรับเก็บข้อมูลลำดับตอนภายในการมอบหมายงาน
    \item \textbf{Export Grades} ตารางสำหรับเก็บข้อมูลการส่งออกเกรด
    \item \textbf{Rubrics} ตารางสำหรับเก็บข้อมูลเกณฑ์การประเมินของใบงาน
    \item \textbf{Bounding Boxes} ตารางสำหรับเก็บข้อมูลกรอบของแต่ละใบงาน
    \item \textbf{Submissions} ตารางสำหรับเก็บข้อมูลการส่งใบงาน หรืออัพโหลดใบงาน
    \item \textbf{Submission Boxes} ตารางสำหรับเก็บข้อมูลกล่องของแต่ใบงาน ที่ใช้ในการทำ Optical Character Recognition (OCR)
    \item \textbf{Uploads} ตารางสำหรับเก็บข้อมูลการอัปโหลดไฟล์ใบงาน
    \item \textbf{Grades} ตารางสำหรับเก็บข้อมูลการให้คะแนน
  \end{enumerate}

\section{โครงสร้างระบบจัดเก็บข้อมูลแบบอ็อบเจ็กต์ (Object Storage)}
  \qquad โครงสร้างระบบจัดเก็บข้อมูลแบบอ็อบเจ็กต์ (Object Storage) ที่ใช้ในโครงงานนี้ คือ MinIO โดยมีการออกแบบโครงสร้างเพื่อรองรับการจัดเก็บไฟล์ต่างๆ
  ที่เกี่ยวข้องกับการทำงานของแพลตฟอร์ม โดยทางผู้พัฒนาได้ออกแบบไว้ในรูปแบบของ Bucket ดังนี้
  \subsection{Buckets} เป็นโครงสร้างหลักที่ใช้ในการจัดเก็บข้อมูลไฟล์ต่างๆ โดยในโครงงานนี้ได้สร้าง Bucket หลักๆ ดังนี้
    \begin{enumerate}
      \item Bucket/CourseID/AssignmentID/: ใช้สำหรับจัดเก็บไฟล์ของใบงาน ไฟล์เพิ่มเติมของใบงาน
      ที่เกี่ยวข้องกับกระบวนวิชาและการมอบหมายงานนั้นๆ 
      \item Bucket/CourseID/AssignmentID/bounding-box/: ใช้สำหรับจัดเก็บไฟล์ที่เกี่ยวข้องกับการกำหนดกรอบ (Bounding Box)
      \item Bucket/CourseID/exports/: ใช้สำหรับจัดเก็บไฟล์ที่เกี่ยวข้องกับการส่งออกเกรดของกระบวนวิชา
    \end{enumerate}

\section{Frontend}
  \qquad Frontend เป็นส่วนที่ผู้ใช้จะโต้ตอบด้วยผ่านทางเว็บเบราว์เซอร์ โดยมีหน้าที่หลักในการแสดงผลข้อมูลและรับคำสั่งจากผู้ใช้
  ซึ่งในโครงงานนี้ได้ใช้ Next.js เป็นเฟรมเวิร์กหลักในการพัฒนา และใช้ Tailwind CSS ร่วมกับ MantineUI ในการออกแบบส่วนติดต่อผู้ใช้ (User Interface)

  \section{การใช้งานพื้นฐาน}
    \qquad แพลตฟอร์มนี้ถูกออกแบบมาให้มี User Interface สำหรับผู้ใช้งาน 2 ประเภท ได้แก่ ผู้สอน (Instructor) และนักศึกษา (Student) โดยแต่ละประเภทจะมีหน้าที่และการใช้งานที่แตกต่างกัน ดังนี้
  \subsection{ผู้สอน (Instructor)}
    \par\hspace*{3em} ผู้สอนมีหน้าที่ในการสร้างคอร์สและมอบหมายงานให้กับนักศึกษา รวมถึงการตรวจสอบและให้คะแนนงานที่นักศึกษาส่งมา 
    \begin{enumerate}
        \item Courses Overview ส่วนนี้เป็นส่วนที่ทําหน้าที่ในการแสดงผลหน้าแรกของเว็บไซต์หลังทําการ Login เข้าสู่ระบบ
         \begin{figure}[H]
            \centering
            \includegraphics[width=0.8\textwidth]{image/Approach/Screen/Course-overview.png}
            \caption[Courses Overview]{รูปภาพแสดงหน้าแรกเมื่ออาจารย์ล็อคอิ้นเข้าสุ่ระบบ}
            \label{fig:CoursesOverview}
          \end{figure}
        \item Dashboard ส่วนนี้เป็นส่วนที่ทําหน้าที่ในการแสดงผลภาพรวมของคอร์สที่ผู้สอนได้สร้างขึ้น
          \begin{figure}[H]
            \centering
            \includegraphics[width=0.8\textwidth]{image/Approach/Screen/Course-dashboard.png}
            \caption[Dashboard]{รูปภาพแสดงเมื่อเข้าไปใน คอร์ส}
            \label{fig:Dashboard}
          \end{figure}
        \item Assignments Management ส่วนนี้เป็นส่วนที่ทําหน้าที่ในการจัดการงานที่ผู้สอนได้มอบหมายให้กับนักศึกษา
          \begin{figure}[H]
            \centering
            \includegraphics[width=0.8\textwidth]{image/Approach/Screen/Assignment-management.png}
            \caption[Assignments Management]{รูปภาพแสดงข้อมูลของงานที่ได้มอบหมายในคอร์ส}
            \label{fig:AssignmentsManagement}
          \end{figure}
        \item Roster Management ส่วนนี้เป็นส่วนที่ทําหน้าที่ในการจัดการรายชื่อนักศึกษาที่ลงทะเบียนในคอร์ส
          \begin{figure}[H]
            \centering
            \includegraphics[width=0.8\textwidth]{image/Approach/Screen/Roster-management.png}
            \caption[Roster Management]{รูปภาพแสดงรายชื่อนักศึกษาที่ลงทะเบียนในคอร์ส}
            \label{fig:RosterManagement}
          \end{figure}
        \item Export Grades ส่วนนี้เป็นส่วนที่ทําหน้าที่ในการส่งออกข้อมูลงานที่ได้รับหมอบหมายเป็นไฟล์ CSV 
          \begin{figure}[H]
            \centering
            \includegraphics[width=0.8\textwidth]{image/Approach/Screen/Export-grade.png}
            \caption[Export Grades]{รูปภาพแสดงหน้าส่งออกเกรดเป็นไฟล์ CSV}
            \label{fig:ExportGrades}
          \end{figure}
        \item Edit Outline ส่วนนี้เป็นส่วนที่ทําหน้าที่ในการสร้างโครงร่างของงานที่ผู้สอนจะใช้ในการตรวจ
        \begin{figure}[H]
            \centering
            \includegraphics[width=0.8\textwidth]{image/Approach/Screen/Edit-outline.png}
            \caption[Edit Outline]{รูปภาพแสดงหน้าสร้างโครงร่างของงาน}
            \label{fig:EditOutline}
          \end{figure}
        \item Manage Submissions ส่วนนี้เป็นส่วนที่ทําหน้าที่ในการจัดการจับคู่ใบงานที่ผู้สอนเป็นผู้อัพโหลด
        \begin{figure}[H]
            \centering
            \includegraphics[width=0.8\textwidth]{image/Approach/Screen/Manage-submission.png}
            \caption[Manage Submissions]{รูปภาพแสดงหน้าจัดการจับคู่ใบงานกับรายชื่อนักศึกษา}
            \label{fig:ManageSubmissions}
          \end{figure}
        \item Grade Submissions ส่วนนี้เป็นส่วนที่ทําหน้าที่ในการให้คะแนนงานทีได้หมอบหมายให้กับนักศึกษา
        \begin{figure}[H]
            \centering
            \includegraphics[width=0.8\textwidth]{image/Approach/Screen/Student-matching.png}
            \caption[Grade Submissions]{รูปภาพแสดงหน้าการให้คะแนนงานที่นักศึกษาได้ส่งมา}
            \label{fig:GradeSubmissions}
          \end{figure}
        \item Review Grades ส่วนนี้เป็นส่วนที่ทำหน้าที่แสดงสถิติของงานที่ได้รับหมอบหมายพร้อมรายชื่อของนักศึกษาที่ส่งงาน
        \begin{figure}[H]
            \centering
            \includegraphics[width=0.8\textwidth]{image/Approach/Screen/Review-grade.png}
            \caption[Review Grades]{รูปภาพแสดงหน้าตรวจสอบเกรดของงานที่ได้รับมอบหมาย}
            \label{fig:ReviewGrades}
          \end{figure}
        \item Statistics ส่วนนี้เป็นส่วนที่ทําหน้าที่ในการแสดงผลสถิติของงานที่ได้รับหมอบหมาย 
        \begin{figure}[H]
            \centering
            \includegraphics[width=0.8\textwidth]{image/Approach/Screen/Assignment-statistic.png}
            \caption[Statistics]{รูปภาพแสดงหน้าสถิติของงานที่ได้รับมอบหมาย}
            \label{fig:Statistics}
          \end{figure}
    \end{enumerate}
  \subsection{นักศึกษา (Student)}
    \par\hspace*{3em} นักศึกษามีหน้าที่ในการตรวจสอบและส่งงานที่ได้รับมอบหมาย
    \begin{enumerate}
        \item Dashboard ส่วนนี้เป็นส่วนที่ทําหน้าที่ในการแสดงผลภาพรวมของงานที่นักศึกษาได้รับหมอบหมายไว้
         \begin{figure}[H]
            \centering
            \includegraphics[width=0.8\textwidth]{image/STD/std-dashboard.png}
            \caption[STDDashboard]{รูปภาพแสดงหน้า Dashboard ของนักศึกษา}
            \label{fig:STDDashboard}
          \end{figure}
        \item Courses Overview ส่วนนี้เป็นส่วนที่ทําหน้าที่ในการแสดงผลภาพรวมของคอร์สที่นักศึกษาได้อยู่ในระบบ
          \begin{figure}[H]
            \centering
            \includegraphics[width=0.8\textwidth]{image/STD/std-courseoverview.png}
            \caption[ CoursesOverview]{รูปภาพแสดงหน้า CoursesOverview ของนักศึกษา}
            \label{fig:Courses Overview}
          \end{figure}
        \item Course Dashboard ส่วนนี้เป็นส่วนที่ทําหน้าที่ในการแสดงผลรายละเอียดของคอร์สที่นักศึกษาได้ลงทะเบียนไว้
          \begin{figure}[H]
            \centering
            \includegraphics[width=0.8\textwidth]{image/STD/std-coursedashboard.png}
            \caption[Course Dashboard]{รูปภาพแสดงหน้า Course Dashboard ของนักศึกษา}
            \label{fig:Course Dashboard}
          \end{figure}
    \end{enumerate}
