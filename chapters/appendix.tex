\chapter{คู่มือการติดตั้ง}
\section{Next.js 14 (Frontend)}
  \begin{figure}[!ht]
    \centering
    \includegraphics[width=0.6\textwidth]{image/Appendix/nextjs.png}
    \caption[Next.js 14]{Next.js 14 Framework}
    \label{fig:nextjs_pic}
  \end{figure}
  \FloatBarrier
  \qquad โครงสร้างของโปรเจกต์นี้ประกอบด้วยหลายไฟล์และโฟลเดอร์ที่สําคัญสําหรับการพัฒนาและการตั้งค่าโปรเจกต์
  โดยมีรายละเอียดดังนี้
  \subsection{Main Files and Folders}
    \begin{enumerate}
      \item .dockerignore: ไฟล์ที่ระบุไฟล์หรือโฟลเดอร์ที่ไม่ต้องการให้Docker คัดลอกไปยัง image
      \item .env ไฟล์ที่ใช้เก็บค่าตัวแปรสภาพแวดล้อม (environment variables) สําหรับการตั้งค่าโปรเจกต์ในสภาพแวดล้อมต่างๆ
      \item .eslintrc.json: ไฟล์การตั้งค่า ESLint สําหรับการตรวจสอบและจัดรูปแบบโค้ด
      \item .gitignore: ไฟล์ที่ระบุไฟล์หรือโฟลเดอร์ที่ไม่ต้องการให้Git ติดตาม
      \item docker-compose.yml: ไฟล์การตั้งค่า Docker Compose สําหรับการจัดการ container หลายๆ ตัว
      \item Dockerfile: ไฟล์การตั้งค่า Docker สําหรับการสร้าง Docker image
      \item next-env.d.ts: ไฟล์การตั้งค่า TypeScript สําหรับโปรเจกต์Next.js
      \item next.config.mjs: ไฟล์การตั้งค่า Next.js
      \item package.json: ไฟล์ที่เก็บข้อมูลเกี่ยวกับโปรเจกต์Node.js รวมถึง dependencies และสคริปต์ต่างๆ
      \item postcss.config.mjs: ไฟล์การตั้งค่า PostCSS
      \item README.md: ไฟล์เอกสารสําหรับโปรเจกต
      \item tailwind.config.ts: ไฟล์การตั้งค่า Tailwind CSS
      \item tsconfig.json: ไฟล์การตั้งค่า TypeScript
    \end{enumerate}

    \subsection{Subfolders}
    \begin{enumerate}
       \item .next: โฟลเดอร์ที่เก็บไฟล์ที่ถูกสร้างขึ้นโดย Next.js หลังจากการ build
       \item app: โฟลเดอร์ที่เก็บไฟล์และโฟลเดอร์ที่เกี่ยวข้องกับการทํางานของแอปพลิเคชัน
       \item components: โฟลเดอร์ที่เก็บคอมโพเนนต์ต่างๆ ที่ใช้ในโปรเจกต์
       \item hooks: โฟลเดอร์ที่เก็บ custom hooks 
       \item store: โฟลเดอร์ที่เก็บ store
       \item public: โฟลเดอร์ที่เก็บไฟล์สาธารณะ เช่น รูปภาพ
    \end{enumerate}
\section{Installation Guide}
  \subsection{Clone the Project from GitHub}
    \qquad git clone https://github.com/jirayu123za/cpe-project-final.git
    cd <repository-directory>
  \subsection{Install Dependencies}
    \qquad npm install
  \subsection{Set up Environment Variables}
    \qquad Create a .env file in the root directory and add the necessary environment variables as specified in the .env.example file.
  \subsection{Run the Development Server}
    \qquad npm run dev
  \subsection{Build for Production}
    \qquad npm run build
  \subsection{Start the Production Server}
    \qquad npm start
  \subsection{Build Docker Image (Optional)}
    \qquad docker build -t <image-name> .


\section{Golang Project (Project service)}
  \begin{figure}[!ht]
    \centering
    \includegraphics[width=0.5\textwidth]{image/Appendix/Golang.png}
    \caption[Golang]{Golang Project Structure}
    \label{fig:golang_pic}
  \end{figure}
  \FloatBarrier
  \subsection{Project Structure}
    \qquad ในส่วนข้างล่างนี้เป็นโครงสร้างของโปรเจกต์ และโฟลเดอร์ต่างๆ ที่ใช้ในโปรเจกต์นี้
    \begin{enumerate}
      \item /cmd: โฟลเดอร์ที่เก็บไฟล์หลักของแอปพลิเคชัน
      \item /internal: โฟลเดอร์ที่เก็บโค้ดภายในของแอปพลิเคชัน
      \item /config: โฟลเดอร์ที่เก็บไฟล์การตั้งค่าต่างๆ
      \item /adapter: โฟลเดอร์ที่เก็บโค้ดที่เชื่อมต่อกับบริการภายนอก
      \item /oauth: โฟลเดอร์ที่เก็บโค้ดที่เกี่ยวข้องกับการตรวจสอบสิทธิ์ OAuth 2.0
      \item /response: โฟลเดอร์ที่เก็บโค้ดที่เกี่ยวข้องกับการสร้างและจัดการการตอบสนอง HTTP
      \item /core: โฟลเดอร์ที่เก็บโค้ดหลักของแอปพลิเคชัน
      \item /repositories: โฟลเดอร์ที่เก็บโค้ดที่เกี่ยวข้องกับการเข้าถึงฐานข้อมูล
      \item /services: โฟลเดอร์ที่เก็บโค้ดที่เกี่ยวข้องกับลอจิกของแอปพลิเคชัน
      \item /utils: โฟลเดอร์ที่เก็บโค้ดที่มีฟังก์ชันใช้ในการช่วยเหลือต่างๆ
      \item /workers: โฟลเดอร์ที่เก็บโค้ดที่เกี่ยวข้องกับการประมวลผลแบบอะซิงโครนัส
      \item /database: โฟลเดอร์ที่เก็บไฟล์ที่เกี่ยวข้องกับการตั้งค่าฐานข้อมูล
      \item /models: โฟลเดอร์ที่เก็บโค้ดที่เกี่ยวข้องกับโมเดลข้อมูล
      \item /routes: โฟลเดอร์ที่เก็บโค้ดที่เกี่ยวข้องกับการกำหนดเส้นทาง (routing) ของแอปพลิเคชัน
      \item /storage: โฟลเดอร์ที่เก็บโค้ดที่เกี่ยวข้องกับการตั้งค่าการจัดเก็บข้อมูลแบบออบเจ็กต์
    \end{enumerate}
  \subsection{Installation}
    \qquad ในส่วนข้างล่างนี้เป็นขั้นตอนการติดตั้งโปรเจกต์นี้
    \begin{enumerate}[label=\arabic*., leftmargin=*, itemsep=6pt]
      \item \textbf{ติดตั้ง Go}\\
            ดาวน์โหลดจาก \url{https://go.dev/doc/install}
      \item \textbf{ติดตั้ง Dependencies}\\
    \begin{lstlisting}
    go mod tidy
    \end{lstlisting}
      \item \textbf{ตั้งค่าฐานข้อมูล}
    \begin{itemize}[leftmargin=1.4em, itemsep=2pt]
      \item ตรวจไฟล์ .env ให้ถูกต้อง
      \item สตาร์ทฐานข้อมูล
    \end{itemize}
    \begin{lstlisting}
    docker compose up -d
    \end{lstlisting}
      \item \textbf{รันโปรเจกต์}
    \begin{lstlisting}
    go run cmd/main.go
    \end{lstlisting}
    \end{enumerate}
  
\chapter{\ifenglish Manual\else คู่มือการใช้งานระบบ\fi}
  \qquad ข้างล่างนี้เป็นขั้นตอนการใช้งานระบบแพลตฟอร์มการตรวจใบงานแบบกระดาษ (Paper Grader) โดยแบ่งออกเป็น 2 ส่วนหลัก
  ได้แก่ ส่วนของ\textbf{ผู้สอน (Instructor)} และส่วนของ\textbf{นักเรียน (Student)}
  
\section{ส่วนของผู้สอน (Instructor)}
    \begin{enumerate}
      \item ทําการ Login เข้าสู่ระบบผ่าน CMU Account หรือ Google Account
   \begin{figure}[H]
      \centering
      \includegraphics[width=0.9\textwidth]{Image/Approach/Screen/Assignment-management.png}
      \label{fig:instructor-login}
    \end{figure}
      \item เมื่อทําการ Login สําเร็จ จะเข้าสู่หน้า Course Overview ที่แสดงคอร์สที่เคยสร้างไว้แล้ว หรือ
สามารถสร้างคอร์สใหม่
  \begin{figure}[H]
      \centering
      \includegraphics[width=0.9\textwidth]{Image/Approach/Screen/Assignment-management.png}
      \label{fig:instructor-courseoverview}
    \end{figure}
      \item 
      \item
      \item
      \item
      \item
    \end{enumerate} 
\section{ส่วนของนักเรียน (Student)}
