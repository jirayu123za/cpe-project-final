\chapter{คู่มือการติดตั้ง}
\section{Next.js 14 (Frontend)}
  \includegraphics[width=1\textwidth]{image/appendix/nextjs.png}
  \qquad โครงสร้างของโปรเจกต์นี้ประกอบด้วยหลายไฟล์และโฟลเดอร์ที่สําคัญสําหรับการพัฒนาและการตั้งค่าโปรเจ
กต์ โดยมีรายละเอียดดังนี้:
  \subsection{Main Files and Folders}
    \begin{enumerate}
      \item .dockerignore: ไฟล์ที่ระบุไฟล์หรือโฟลเดอร์ที่ไม่ต้องการให้Docker คัดลอกไปยัง image
      \item .env ไฟล์ที่ใช้เก็บค่าตัวแปรสภาพแวดล้อม (environment variables)สําหรับการตั้งค่าโปรเจกต์ในสภาพแวดล้อมต่างๆ
      \item .eslintrc.json: ไฟล์การตั้งค่า ESLint สําหรับการตรวจสอบและจัดรูปแบบโค้ด
      \item .gitignore: ไฟล์ที่ระบุไฟล์หรือโฟลเดอร์ที่ไม่ต้องการให้Git ติดตาม
      \item docker-compose.yml: ไฟล์การตั้งค่า Docker Compose สําหรับการจัดการ container
หลายๆ ตัว
      \item Dockerfile: ไฟล์การตั้งค่า Docker สําหรับการสร้าง Docker image
      \item next-env.d.ts: ไฟล์การตั้งค่า TypeScript สําหรับโปรเจกต์Next.js
      \item next.config.mjs: ไฟล์การตั้งค่า Next.js
      \item package.json: ไฟล์ที่เก็บข้อมูลเกี่ยวกับโปรเจกต์Node.js รวมถึง dependencies และสคริปต์
ต่างๆ
      \item postcss.config.mjs: ไฟล์การตั้งค่า PostCSS
      \item README.md: ไฟล์เอกสารสําหรับโปรเจกต
      \item tailwind.config.ts: ไฟล์การตั้งค่า Tailwind CSS
      \item tsconfig.json: ไฟล์การตั้งค่า TypeScript
    \end{enumerate}

    \subsection{Subfolders}
    \begin{enumerate}
       \item .next: โฟลเดอร์ที่เก็บไฟล์ที่ถูกสร้างขึ้นโดย Next.js หลังจากการ build
       \item app: โฟลเดอร์ที่เก็บไฟล์และโฟลเดอร์ที่เกี่ยวข้องกับการทํางานของแอปพลิเคชัน
       \item components: โฟลเดอร์ที่เก็บคอมโพเนนต์ต่างๆ ที่ใช้ในโปรเจกต์
       \item hooks: โฟลเดอร์ที่เก็บ custom hooks 
       \item store: โฟลเดอร์ที่เก็บ store
       \item public: โฟลเดอร์ที่เก็บไฟล์สาธารณะ เช่น รูปภาพ
    \end{enumerate}
\section{Installation Guide}
  \subsection{Clone the Project from GitHub}
    \qquad git clone https://github.com/jirayu123za/cpe-project-final.git
cd <repository-directory>
  \subsection{Install Dependencies}
    \qquad npm install
  \subsection{Set up Environment Variables}
    \qquad Create a .env file in the root directory and add the necessary environment variables as specified in the .env.example file.
  \subsection{Run the Development Server}
    \qquad npm run dev
  \subsection{Build for Production}
    \qquad npm run build
  \subsection{Start the Production Server}
    \qquad npm start
  \subsection{Build Docker Image (Optional)}
    \qquad npm start

\section{Golang Project (Project service)}
  \qquad รูป
  \subsection{Project Structure}
    \qquad ในส่วนข้างล่างนี้เป็นโครงสร้างของโปรเจกต์ และโฟลเดอร์ต่างๆ ที่ใช้ในโปรเจกต์นี้
    \begin{enumerate}
      \item 
      \item 
      \item 
      \item
      \item
      \item
      \item
    \end{enumerate}
  \subsection{Installation}
    \qquad ในส่วนข้างล่างนี้เป็นขั้นตอนการติดตั้งโปรเจกต์นี้
    \begin{enumerate}
      \item 
      \item 
      \item 
      \item
      \item
      \item
      \item
    \end{enumerate}
  





Text for a section in the first appendix goes here.

test ทดสอบฟอนต์ serif ภาษาไทย

\textsf{test ทดสอบฟอนต์ sans serif ภาษาไทย}

\verb+test ทดสอบฟอนต์ teletype ภาษาไทย+

\texttt{test ทดสอบฟอนต์ teletype ภาษาไทย}

\textbf{ตัวหนา serif ภาษาไทย \textsf{sans serif ภาษาไทย} \texttt{teletype ภาษาไทย}}

\textit{ตัวเอียง serif ภาษาไทย \textsf{sans serif ภาษาไทย} \texttt{teletype ภาษาไทย}}

\textbf{\textit{ตัวหนาเอียง serif ภาษาไทย \textsf{sans serif ภาษาไทย} \texttt{teletype ภาษาไทย}}}

\url{https://www.example.com/test_ทดสอบ_url}

\chapter{\ifenglish Manual\else คู่มือการใช้งานระบบ\fi}
  \qquad ข้างล่างนี้เป็นขั้นตอนการใช้งานระบบแพลตฟอร์มการตรวจใบงานแบบกระดาษ (Paper Grader) โดยแบ่งออกเป็น 2 ส่วนหลัก
  ได้แก่ ส่วนของ\textbf{ผู้สอน (Instructor)} และส่วนของ\textbf{นักเรียน (Student)}
  
\section{ส่วนของผู้สอน (Instructor)}
\section{ส่วนของนักเรียน (Student)}
