\chapter{คู่มือการติดตั้ง}
\section{Next.js 14 (Frontend)}
  \begin{figure}[!ht]
    \centering
    \includegraphics[width=0.6\textwidth]{image/Appendix/nextjs.png}
    \caption[Next.js 14]{Next.js 14 Framework}
    \label{fig:nextjs_pic}
  \end{figure}
  \FloatBarrier
  \qquad โครงสร้างของโปรเจกต์นี้ประกอบด้วยหลายไฟล์และโฟลเดอร์ที่สําคัญสําหรับการพัฒนาและการตั้งค่าโปรเจกต์
  โดยมีรายละเอียดดังนี้
  \subsection{Main Files and Folders}
    \begin{enumerate}
      \item .dockerignore: ไฟล์ที่ระบุไฟล์หรือโฟลเดอร์ที่ไม่ต้องการให้Docker คัดลอกไปยัง image
      \item .env ไฟล์ที่ใช้เก็บค่าตัวแปรสภาพแวดล้อม (environment variables) สําหรับการตั้งค่าโปรเจกต์ในสภาพแวดล้อมต่างๆ
      \item .eslintrc.json: ไฟล์การตั้งค่า ESLint สําหรับการตรวจสอบและจัดรูปแบบโค้ด
      \item .gitignore: ไฟล์ที่ระบุไฟล์หรือโฟลเดอร์ที่ไม่ต้องการให้Git ติดตาม
      \item docker-compose.yml: ไฟล์การตั้งค่า Docker Compose สําหรับการจัดการ container หลายๆ ตัว
      \item Dockerfile: ไฟล์การตั้งค่า Docker สําหรับการสร้าง Docker image
      \item next-env.d.ts: ไฟล์การตั้งค่า TypeScript สําหรับโปรเจกต์Next.js
      \item next.config.mjs: ไฟล์การตั้งค่า Next.js
      \item package.json: ไฟล์ที่เก็บข้อมูลเกี่ยวกับโปรเจกต์Node.js รวมถึง dependencies และสคริปต์ต่างๆ
      \item postcss.config.mjs: ไฟล์การตั้งค่า PostCSS
      \item README.md: ไฟล์เอกสารสําหรับโปรเจกต
      \item tailwind.config.ts: ไฟล์การตั้งค่า Tailwind CSS
      \item tsconfig.json: ไฟล์การตั้งค่า TypeScript
    \end{enumerate}

    \subsection{Subfolders}
    \begin{enumerate}
       \item .next: โฟลเดอร์ที่เก็บไฟล์ที่ถูกสร้างขึ้นโดย Next.js หลังจากการ build
       \item app: โฟลเดอร์ที่เก็บไฟล์และโฟลเดอร์ที่เกี่ยวข้องกับการทํางานของแอปพลิเคชัน
       \item components: โฟลเดอร์ที่เก็บคอมโพเนนต์ต่างๆ ที่ใช้ในโปรเจกต์
       \item hooks: โฟลเดอร์ที่เก็บ custom hooks 
       \item store: โฟลเดอร์ที่เก็บ store
       \item public: โฟลเดอร์ที่เก็บไฟล์สาธารณะ เช่น รูปภาพ
    \end{enumerate}
\section{Installation Guide}
  \subsection{Clone the Project from GitHub}
    \qquad git clone https://github.com/jirayu123za/cpe-project-final.git
    cd <repository-directory>
  \subsection{Install Dependencies}
    \qquad npm install
  \subsection{Set up Environment Variables}
    \qquad Create a .env file in the root directory and add the necessary environment variables as specified in the .env.example file.
  \subsection{Run the Development Server}
    \qquad npm run dev
  \subsection{Build for Production}
    \qquad npm run build
  \subsection{Start the Production Server}
    \qquad npm start
  \subsection{Build Docker Image (Optional)}
    \qquad docker build -t <image-name> .


\section{Golang Project (Project service)}
  \begin{figure}[!ht]
    \centering
    \includegraphics[width=0.5\textwidth]{image/Appendix/Golang.png}
    \caption[Golang]{Golang Project Structure}
    \label{fig:golang_pic}
  \end{figure}
  \FloatBarrier
  \subsection{Project Structure}
    \qquad ในส่วนข้างล่างนี้เป็นโครงสร้างของโปรเจกต์ และโฟลเดอร์ต่างๆ ที่ใช้ในโปรเจกต์นี้
    \begin{enumerate}
      \item /cmd: โฟลเดอร์ที่เก็บไฟล์หลักของแอปพลิเคชัน
      \item /internal: โฟลเดอร์ที่เก็บโค้ดภายในของแอปพลิเคชัน
      \item /config: โฟลเดอร์ที่เก็บไฟล์การตั้งค่าต่างๆ
      \item /adapter: โฟลเดอร์ที่เก็บโค้ดที่เชื่อมต่อกับบริการภายนอก
      \item /oauth: โฟลเดอร์ที่เก็บโค้ดที่เกี่ยวข้องกับการตรวจสอบสิทธิ์ OAuth 2.0
      \item /response: โฟลเดอร์ที่เก็บโค้ดที่เกี่ยวข้องกับการสร้างและจัดการการตอบสนอง HTTP
      \item /core: โฟลเดอร์ที่เก็บโค้ดหลักของแอปพลิเคชัน
      \item /repositories: โฟลเดอร์ที่เก็บโค้ดที่เกี่ยวข้องกับการเข้าถึงฐานข้อมูล
      \item /services: โฟลเดอร์ที่เก็บโค้ดที่เกี่ยวข้องกับลอจิกของแอปพลิเคชัน
      \item /utils: โฟลเดอร์ที่เก็บโค้ดที่มีฟังก์ชันใช้ในการช่วยเหลือต่างๆ
      \item /workers: โฟลเดอร์ที่เก็บโค้ดที่เกี่ยวข้องกับการประมวลผลแบบอะซิงโครนัส
      \item /database: โฟลเดอร์ที่เก็บไฟล์ที่เกี่ยวข้องกับการตั้งค่าฐานข้อมูล
      \item /models: โฟลเดอร์ที่เก็บโค้ดที่เกี่ยวข้องกับโมเดลข้อมูล
      \item /routes: โฟลเดอร์ที่เก็บโค้ดที่เกี่ยวข้องกับการกำหนดเส้นทาง (routing) ของแอปพลิเคชัน
      \item /storage: โฟลเดอร์ที่เก็บโค้ดที่เกี่ยวข้องกับการตั้งค่าการจัดเก็บข้อมูลแบบออบเจ็กต์
    \end{enumerate}
  \subsection{Installation}
    \qquad ในส่วนข้างล่างนี้เป็นขั้นตอนการติดตั้งโปรเจกต์นี้
    \begin{enumerate}[label=\arabic*., leftmargin=*, itemsep=6pt]
      \item \textbf{ติดตั้ง Go}\\
            ดาวน์โหลดจาก \url{https://go.dev/doc/install}
      \item \textbf{ติดตั้ง Dependencies}\\
    \begin{lstlisting}
    go mod tidy
    \end{lstlisting}
      \item \textbf{ตั้งค่าฐานข้อมูล}
    \begin{itemize}[leftmargin=1.4em, itemsep=2pt]
      \item ตรวจไฟล์ .env ให้ถูกต้อง
      \item สตาร์ทฐานข้อมูล
    \end{itemize}
    \begin{lstlisting}
    docker compose up -d
    \end{lstlisting}
      \item \textbf{รันโปรเจกต์}
    \begin{lstlisting}
    go run cmd/main.go
    \end{lstlisting}
    \end{enumerate}
  
\chapter{\ifenglish Manual\else คู่มือการใช้งานระบบ\fi}
  \qquad ข้างล่างนี้เป็นขั้นตอนการใช้งานระบบแพลตฟอร์มการตรวจใบงานแบบกระดาษ (Paper Grader) โดยแบ่งออกเป็น 2 ส่วนหลัก
  ได้แก่ ส่วนของ\textbf{ผู้สอน (Instructor)} และส่วนของ\textbf{นักเรียน (Student)}
  
\section{ส่วนของผู้สอน (Instructor)}
    \begin{enumerate}
      \item ทําการ Login เข้าสู่ระบบผ่าน CMU Account หรือ Google Account 
   \begin{figure}[H]
      \centering
      \includegraphics[width=0.4\textwidth]{Image/modal/login.png}
      \label{fig:instructor-login}
    \end{figure}
      \item เมื่อทําการ Login สําเร็จ จะเข้าสู่หน้า Course Overview ที่แสดงคอร์สที่เคยสร้างไว้แล้ว หรือ
สร้างคอร์สใหม่ผ่านปุ่ม "Create course"
  \begin{figure}[H]
      \centering
      \includegraphics[width=0.8\textwidth]{Image/Approach/Screen/Course-overview.png}
      \caption[courseoverview]{รูปภาพแสดงหน้า Course Overview เมื่อเข้ามาในระบบของผู้สอน}
      \label{fig:instructor-courseoverview} 
      \includegraphics[width=0.45\textwidth]{Image/modal/instructor-create-course.png}
      \caption[createassignment]{รูปภาพแสดงการสร้าง Course ใหม่}
      \label{fig:createassignment}
      \end{figure}
      \item หน้าการจัดการงานที่ได้หมอบหมาย (Assignment Management) โดยผู้สอนสามารถตั้งเวลาส่งงาน วันหมดเขต วันตัดรอบ ปล่อยเกรด และอื่นๆได้ 
      \begin{figure}[H]
      \centering
      \includegraphics[width=0.9\textwidth]{Image/Approach/Screen/Assignment-management.png}
      \caption[Assignment]{รูปภาพแสดงการหน้าการจัดการงานที่ได้หมอบหมาย (Assignment Management)}
      \includegraphics[width=0.6\textwidth]{Image/modal/instructor-editassignment.png}
      \caption[settingAssignment]{รูปภาพแสดงการตั้งค่างานที่มอบหมาย}
      \includegraphics[width=0.6\textwidth]{Image/modal/instructor-settingtime.png}
      \caption[settingtime]{รูปภาพแสดงการตั้งเวลาสำหรับงานที่มอบหมาย}
      \end{figure}
     \item หน้าการจัดการรายชื่อภายในคอร์ส (Roster) จะแสดงรายชื่อภายในคอร์ส
\begin{figure}[H]
  \centering
  \includegraphics[width=0.9\textwidth]{Image/Approach/Screen/Roster-management.png}
  \caption{รูปภาพแสดงหน้าการจัดการรายชื่อ (Roster)}
  \label{fig:roster-management}
\end{figure}

\begin{enumerate}
  \item สามารถเพิ่มรายชื่อได้ 2 รูปแบบ โดยรูปแบบแรกเป็นการเพิ่มแบบ \textit{single user}
  \begin{figure}[H]
    \centering
    \includegraphics[width=0.6\textwidth]{Image/modal/instructor-addsingle.png}
    \caption{หน้าการเพิ่มรายชื่อรายบุคคลเข้าคอร์ส (Single)}
    \label{fig:add-single}
  \end{figure}

  \item รูปแบบที่สองเป็นการเพิ่มจากไฟล์ โดยมีเทมเพลตให้เลือก 2 แบบ
  (ไฟล์ Excel จากสำนักทะเบียน และไฟล์รูปแบบของแพลตฟอร์ม)
  \begin{figure}[H]
    \centering
    \includegraphics[width=0.9\textwidth]{Image/modal/instructor-addcsv01.png}
    \caption{หน้าการเลือก template ในการอัปโหลดไฟล์ (File Import)}
    \label{fig:add-file}
  \end{figure}
  \renewcommand{\labelenumii}{(\roman{enumii})}
    \begin{enumerate}
      \item หลังจากเลือก template แล้ว จะเข้าสู่ขั้นตอนที่สอง คือขั้นตอนการอัปโหลดไฟล์รายชื่อนักศึกษา
    \begin{figure}[H]
    \centering
    \includegraphics[width=0.9\textwidth]{Image/modal/instructor-addcsv02.png}
    \caption{หน้าการอัปโหลดไฟล์ template (File Import)}
    \label{fig:add-file2}
    \end{figure}
    \item หลังจากอัปโหลดไฟล์แล้ว จะเข้าสู่ขั้นตอนที่สาม คือขั้นตอนการจับคู่คอลัมน์ของไฟล์ที่อัปโหลดเข้ากับตารางในแพลตฟอร์ม
    \begin{figure}[H]
    \centering
    \includegraphics[width=0.9\textwidth]{Image/modal/instructor-addcsv03.png}
    \caption{หน้าการจับคู่คอลัมน์ (File Import)}
    \label{fig:add-file3}
    \end{figure}
    \end{enumerate}
\end{enumerate}

      \item Edit Outline เมื่อทำการสร้างงานที่ได้หมอบหมาย (Assignments) แล้วสามารถเข้าไปที่งานที่ได้หมอบหมาย (Assignments) เพื่อสร้างแม่แบบคำถาม (Bounding box) และสร้างเกณฑ์การให้คะแนน (Rubric) เพื่อใช้ในการตรวจ
       \renewcommand{\labelenumii}{(\roman{enumii})}
    \begin{enumerate}
      \item หน้าการสร้างแม่แบบคำถาม (Bounding box) ที่เอาไว้ใช้ในตอนตรวจงานที่ได้มอบหมายของนักศึกษา
    \begin{figure}[H]
    \centering
    \includegraphics[width=0.9\textwidth]{Image/Approach/Screen/Edit-outline.png}
    \caption{หน้าการสร้างแม่แบบคำถาม (Bounding box)}
    \label{fig:Bounding box}
    \end{figure}
    \item หลังจากการสร้างแม่แบบคำถามเสร็จแล้ว ส่วนต่อไปจะเป็นการสร้างเกณฑ์การให้คะแนน (Rubric) ของคำถามแต่ละข้อ
    \begin{figure}[H]
    \centering
    \includegraphics[width=0.9\textwidth]{Image/Approach/Screen/Edit-rubric.png}
    \caption{หน้าการสร้างแม่แบบคำถามจับคู่คอลัมน์ (Rubric)}
    \label{fig:Rubric}
    \end{figure}
    \end{enumerate}
      \item Management submission เป็นหน้าการจัดอัปโหลดไฟล์ของนักศึกษาเมื่องานที่ได้มอบหมายถูกตั้งแบบอาจารย์เป็นผู้ส่ง 
         \renewcommand{\labelenumii}{(\roman{enumii})}
    \begin{enumerate}
      \item ขั้นตอนแรกจะเป็นการอัปโหลดไฟล์ที่รวมใบงานของนักศึกษาไว้ในไฟล์เดียว โดยทางระบบจะทำการแยกไฟล์ให้โดยอัตโนมัติ
    \begin{figure}[H]
    \centering
    \includegraphics[width=0.9\textwidth]{Image/Approach/Screen/Manage-submission.png}
    \caption{หน้าการอัปโหลดไฟล์ (Manage-submission)}
    \label{fig:Manage-submission}
    \end{figure}
    \item ขั้นตอนที่สองหลังจากอัปโหลดไฟล์แล้วจะเป็นการจับคู่ไฟล์กับรายชื่อนักศึกษาภายในคอร์ส โดยสามารถใช้ระบบ OCR ที่จะจับคู่ให้อัตโนมัติ หรือจะจับคู่เอง 
    \begin{figure}[H]
    \centering
    \includegraphics[width=0.9\textwidth]{Image/Approach/Screen/Manage-submission2.png}
    \caption{หน้าการจับคู่ไฟล์กับรายชื่อนักศึกษา (Manage-submission)}
    \label{fig:Manage-submission2}
    \end{figure}
    \end{enumerate}
      \item Grade Submission รายชื่อนักศึกษาและ คำถาม เพื่อเลือกเข้าไปตรวจให้คะแนน
      \begin{enumerate}
  \item ส่วนแรกที่เข้ามาจะเป็นแสดงรายชื่อนักศึกษา 
  \begin{figure}[H]
    \centering
    \includegraphics[width=0.9\textwidth]{Image/Approach/Screen/Grade-submission.png}
    \caption{หน้าการแสดงรายชื่อนักศึกษา (Grade-submission)}
    \label{fig:Grade-submission}
  \end{figure}
    \begin{itemize}
    \item สามารถคลิ๊กเพื่อเข้าได้ดูไฟล์ที่นักศึกษาคนนั้นๆ ส่งมาได้
    \begin{figure}[H]
    \centering
    \includegraphics[width=0.9\textwidth]{Image/Approach/Screen/Grade-submission2.png}
    \caption{หน้าแสดงไฟล์ของนักศึกษา (Grade submission review)}
    \label{fig:Grade-submission-review}
  \end{figure}
  \end{itemize}

  \item ส่วนที่สองจะแสดงเป็นคำถามโดยสามารถกดเลือกที่คำถามเพื่อเข้าไปตรวจหรือกดเลือกที่ submission เพิ่อเลือก submission ที่จะเข้าไปตรวจ 
  \begin{figure}[H]
    \centering
    \includegraphics[width=0.9\textwidth]{Image/Approach/Screen/Grade-submission-question.png}
    \caption{หน้าการเลือกคำถามเพื่อเข้าไปตรวจ (Grade submission question)}
    \label{fig:Grade-submission-question}
  \end{figure}
    \begin{itemize}
    \item สามารถกดเลิอกที่ submission เพิ่มเปิดหน้าเลือก submission ที่จะเข้าไปตรวจได้
    \begin{figure}[H]
    \centering
    \includegraphics[width=0.9\textwidth]{Image/Approach/Screen/Grade-submission-question-sub.png}
    \caption{หน้าแสดงรายชื่อของนักศึกษา (Grade question submission )}
    \label{fig:Grade-submission-question-sub}
  \end{figure}
\end{itemize}
  \item หลังจากกดเลือกคำถามหรือ submission ที่ต้องการจะตรวจแล้วจะมาสู่หน้าการตรวจ โดยหน้าการตรวจนั้น สามารถแก้ไขเกณฑ์การให้คะแนน (Rubric) และสามารถใช้ปุ่มลัด (Shotcut key) เลข 1-9 เพิ่อใช้ในการเลือกเกณฑ์การให้คะแนน (Rubric) อีกทั้งยังสามารถ ใช้ ลูกศรขึ้นหรือลงในการเปลี่ยนคำถาม และยังสามารถใช้ลูกศรซ้ายหรือขวาในการเปลี่ยน submission ได้
  \begin{figure}[H]
    \centering
    \includegraphics[width=0.9\textwidth]{Image/Approach/Screen/Grade.png}
    \caption{หน้าการตรวจให้คะแนน (Grade)}
    \label{fig:Grade}
  \end{figure}
   \end{enumerate} 
      \item Review Grade เป็นหน้าที่จะแสดงข้อมูลสถิติของงานที่ได้มอบหมาย  นั้นๆ เป็นกราฟที่สามารถปรับ No. of bin ได้ และในส่วนล่างเป็นตารางรายชื่อของนักศึกษา แสดงคะแนน พร้อมสถานะ ต่างๆ
    \begin{figure}[H]
    \centering
    \includegraphics[width=0.9\textwidth]{Image/Approach/Screen/Review-grade.png}
    \caption{หน้าการแสดงสถิติ (Review grade)}
    \label{fig:Review grade}
    \end{figure}
      \item Statistic เป็นหน้าที่จะแสดงข้อมูลสถิติของงานที่ได้มอบหมาย สามารถกดเลิอกที่งานที่ได้มอบหมายเพื่อเลือกดูสถิติของงานที่ได้มอบหมายต่างๆ ใน Course ได้ และยังสามารถเลือก section เพื่อดูสถิติได้สามารถเลือกดูของ section นั้นๆ หรือสามารถดูของหลายๆ section พร้อมกันได้ และในส่วนด้านล่างจะเป็นตารางแสดงคำถาม สามารถกดเลือกเพื่อดู สถิติของเกณฑ์การให้คะแนน (Rubric) ได้
    \begin{figure}[H]
    \centering
    \includegraphics[width=0.9\textwidth]{Image/Approach/Screen/Assignment-statistic.png}
    \caption{หน้าการแสดงสถิติของงานที่ได้มอบหมาย (Assignment statistic)}
    \label{fig:Assignment-statistic}
    \begin{itemize}
    \item สามารถกดเลิอกที่คำถาม เพิ่มเปิดหน้าต่างแสดงสถิติของเกณฑ์การให้คะแนน (Rubric)
    \begin{figure}[H]
    \centering
    \includegraphics[width=0.6\textwidth]{Image/modal/Rubric.png}
    \caption{หน้าแสดงสถิติของเกณฑ์การให้คะแนน (Rubric statistic )}
    \label{fig:Rubric statistic}
    \end{figure}
    \end{itemize}
    \end{figure}
    \item Data Export เป็นหน้าแสดงประวัติการ Export ไฟล์สามารถกด download ไฟล์หรือลบไฟล์ได้ อีกทั้งยังสามารถกดเลือกไฟล์ที่จะ export ข้อมูลไปไฟล์ excel เพื่อนำไปใช้ต่อได้
    \begin{figure}[H]
    \centering
    \includegraphics[width=0.9\textwidth]{Image/Approach/Screen/Export-grade.png}
    \caption{หน้าการแสดงประวัติการ Export (Export grade)}
    \label{fig:Export-grade}
    \begin{itemize}
    \item สามารถกดเลิอกที่ Export เพื่อเลือกงานที่ได้มอบหมาย ที่ต้องการ Export ได้
    \begin{figure}[H]
    \centering
    \includegraphics[width=0.6\textwidth]{Image/modal/instructor-export.png}
    \caption{หน้าแสดงการเลือกงานที่ได้มอบหมายเพื่อ Export ( Export )}
    \label{fig:instructor-export}
    \end{figure}
    \end{itemize}
    \end{figure}
    \end{enumerate} 


\section{ส่วนของนักศึกษา (Student)}
\renewcommand{\labelenumii}{\theenumi.\arabic{enumii}}
\begin{enumerate}
  \item หน้า Dashboard แสดงให้เห็นถึงงานที่ได้รับหมอบหมายโดยเรียงจากงานที่ไกล้ครบกำหนดไปจนถึงงานที่ยังไกลวันครบกำหนด อีกทั้งยังสามารถดูงานที่ได้รับหมอบหมายที่เลยเวลาที่กำหนดได้ในแทบที่สอง
    \begin{figure}[H]
      \centering
      \includegraphics[width=0.92\textwidth]{image/STD/std-dashboard.png}
      \caption{หน้า Dashboard แสดงงานที่ได้รับหมอบหมายที่อยู่ภายในเวลาที่กำหนด (Dashboard )}
      \label{fig:std-dashboard-1}
    \end{figure}
    \begin{figure}[H]
      \centering
      \includegraphics[width=0.92\textwidth]{image/STD/std-dashboardover.png}
      \caption{หน้า Dashboard แสดงงานที่ได้รับหมอบหมายมราเลยเวลาที่กำหนด(Dashboard )}
      \label{fig:std-dashboard-2}
    \end{figure}
    \FloatBarrier

   
    \begin{enumerate}
      \item หน้าการส่งงาน เมื่อคลิ๊กเลือกที่ icon จะสามารถดูไฟล์เพิ่มเติมที่อาจารย์ได้แนบมาพร้อมไฟล์ใบงาน และ นักศึกษาสามารถส่งงานได้
        \begin{figure}[H]
          \centering
          \includegraphics[width=0.7\textwidth]{image/modal/std-submitwork.png}
          \caption{หน้าการส่งงานของนักศึกษา}
          \label{fig:std-submit}
        \end{figure}
        \FloatBarrier
    \end{enumerate}

 
  \item หน้า course overview เป็นหน้าแสดงรายการของคอร์สที่นักศึกษาอยู่ทั้งหมด
    \begin{figure}[H]
      \centering
      \includegraphics[width=0.92\textwidth]{image/STD/std-courseoverview.png}
      \caption{หน้า Course Overview ของนักศึกษา}
      \label{fig:std-course-overview}
    \end{figure}
    \FloatBarrier
  \item หน้า Course Assignmen เมื่อคลิ๊กเข้าไปในคอร์ส จะเป็นหน้าแสดงรายการงานที่ได้หมอบหมายภายในคอร์สนั้นๆ 
    \begin{figure}[H]
      \centering
      \includegraphics[width=0.92\textwidth]{image/STD/std-coursedashboard.png}
      \caption{หน้า Course Assignment ของนักศึกษา}
      \label{fig:std-course-assignment}
    \end{figure}
    \FloatBarrier

   
    \begin{enumerate}
      \item หน้า review assignment เมื่อกดเลือกที่งานที่ได้รับหมอบหมายจะไปสู่หน้าการ review assignment จะแสดงไฟล์ที่เราส่งและทางฝั่งขวา จะแสดงให้เห็นว่ามีคำถามข้อไหนบ้างมีคะแนนเต็มเท่าไหร่
        \begin{figure}[H]
          \centering
          \includegraphics[width=0.92\textwidth]{image/STD/std-reviewgrade.png}
          \caption{หน้า Review Assignment ของนักศึกษา}
          \label{fig:std-review-assignment}
        \end{figure}
        \FloatBarrier
    \end{enumerate}

\end{enumerate}
