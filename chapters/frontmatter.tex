\maketitle
\makesignature

\ifproject
\begin{abstractTH}
\sloppy \par
โครงงานนี้มีจุดประสงค์เพื่อพัฒนาแพลตฟอร์มสำหรับการตรวจใบงานแบบกระดาษในมหาวิทยาลัยเชียงใหม่ โดยแพลตฟอร์มที่พัฒนาขึ้นจะช่วยลดปัญหาที่เกิดจากกระบวนการตรวจและบันทึกคะแนนในปัจจุบัน ซึ่งเป็นกระบวนการที่ใช้เวลานานและซับซ้อน โดยเฉพาะการเปลี่ยนแปลงเกณฑ์คะแนนที่ต้องมีการคำนวณใหม่ซ้ำซ้อน แพลตฟอร์มที่พัฒนาจะมีฟังก์ชันที่ช่วยในการกำหนดเกณฑ์คะแนนที่ยืดหยุ่น รองรับการปรับเปลี่ยนได้ตามต้องการ รวมถึงการจัดการข้อมูลคะแนนอย่างเป็นระบบและสามารถติดตามผลการตรวจงานได้อย่างมีประสิทธิภาพ นอกจากนี้แพลตฟอร์มนี้ยังเป็นเครื่องมือสำหรับนักศึกษาในการติดตามคะแนนของใบงานและยื่นคำร้องขอตรวจคะแนนใหม่ได้เมื่อพบความผิดพลาด ผลจากการพัฒนาระบบนี้จะช่วยลดภาระการจัดเก็บและค้นหาใบงานในรูปแบบกระดาษของอาจารย์ รวมถึงลดความล่าช้าและความผิดพลาดจากการตรวจและบันทึกคะแนนในรูปแบบดั้งเดิม ทำให้การจัดการเอกสารและคะแนนเป็นไปอย่างมีประสิทธิภาพมากยิ่งขึ้น
\end{abstractTH}

\begin{abstract}
\par
The objective of this project is to develop a platform for grading paper-based assignments at Chiang Mai University. The proposed platform aims to address challenges in the current grading and score-recording processes, which are time-consuming and complex, particularly when changes to grading criteria require recalculations. The developed platform will feature flexible grading functions that accommodate adjustments as needed, along with systematic score management to efficiently track grading outcomes. Additionally, this platform serves as a tool for students to monitor their assignment scores and submit requests for regrading when errors are identified. The implementation of this system will reduce the burden of paper-based assignment storage and retrieval for instructors, decrease delays in the grading process, and minimize errors associated with traditional scoring and record-keeping. As a result, it will enhance the efficiency of document and score management.
\end{abstract}

\iffalse
\begin{dedication}
This document is dedicated to all Chiang Mai University students.

Dedication page is optional.
\end{dedication}
\fi % \iffalse

\begin{acknowledgments}
Your acknowledgments go here. Make sure it sits inside the
\texttt{acknowledgment} environment.

\acksign{2020}{5}{25}
\end{acknowledgments}%
\fi % \ifproject

\contentspage

\ifproject
\figurelistpage

\tablelistpage
\fi % \ifproject

% \abbrlist % this page is optional

% \symlist % this page is optional

% \preface % this section is optional
