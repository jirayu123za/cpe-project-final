\chapter{\ifproject%
\ifenglish Experimentation and Results\else การทดลองและผลลัพธ์\fi
\else%
\ifenglish System Evaluation\else การประเมินระบบ\fi
\fi}

ในบทนี้จะทําการทดสอบการทํางานของระบบในฟังก์ชันหลักต่างๆ

\section{การทดสอบฟังก์ชันการทํางานของระบบ}
  \qquad ในการทดสอบฟังก์ชันการทํางานของระบบแพลตฟอร์มการตรวจใบงานแบบกระดาษ (Paper Grader) ได้มีการทดสอบฟังก์ชัน
  หลักต่างๆ ของระบบเพื่อให้แน่ใจว่าระบบสามารถทํางานได้ตามที่ออกแบบไว้ โดยการทดสอบจะครอบคลุมฟังก์ชันหลักในแต่ละส่วนของผู้ใช้งานของแต่ละบทบาท
  ได้แก่ ส่วนของ \textbf{ผู้สอน (Instructor)} และส่วนของ \textbf{นักศึกษา (Student)} โดยมีรายละเอียดดังนี้
  \subsection{การทดสอบความถูกต้องของการสมัครสมาชิกและการเข้าสู่ระบบของแพลตฟอร์ม}
    \qquad ในการทดสอบความถูกต้องของระบบแพลตฟอร์มการตรวจใบงานแบบกระดาษ (Paper Grader)
    ได้มีการทดสอบฟังก์ชันดังนี้
    \subsubsection{การเข้าใช้งานครั้งแรกผ่าน CMU Account}
      \begin{enumerate}
        \item \textbf{วัตถุประสงค์:} เพื่อทดสอบว่าผู้ใช้งานสามารถเข้าสู่ระบบผ่าน CMU Account ได้อย่างถูกต้อง
        \item \textbf{ขั้นตอนการทดสอบ:}
          \begin{enumerate}
            \item เปิดหน้าเข้าสู่ระบบของแพลตฟอร์ม
            \item คลิกที่ปุ่ม "Sign-In"
            \item เลือกประเภทบัญชีเป็น "CMU Account"
            \item ป้อนข้อมูล CMU Account ที่ถูกต้อง
            \item ป้อนข้อมูลการยืนยันตัวตนสองขั้นตอน (ถ้ามี)
            \item คลิกที่ปุ่ม "Login"
            \item รอให้ระบบทําการตรวจสอบและเข้าสู่ระบบ
            \item กรอกรายละเอียดการใช้งานครั้งแรก ตามที่แพลตฟอร์มร้องขอ
            \item คลิกที่ปุ่ม "Sign Up" เพื่อสร้างบัญชีผู้ใช้งานใหม่
            \item ผู้ใช้งานถูกนําส่งไปยังหน้าแดชบอร์ดของแต่ละบทบาท
          \end{enumerate}
        \item \textbf{ผลลัพธ์ที่คาดหวัง:} ผู้ใช้งานสามารถเข้าสู่ระบบผ่าน CMU Account ได้อย่างถูกต้อง
        และถ้าเป็นการเข้าสู่ระบบครั้งแรกจะถูกนําไปยังหน้ากรอกรายละเอียดการใช้งานครั้งแรก และหลังจากกรอกข้อมูลครบถ้วน
        จะถูกนําส่งไปยังหน้าแดชบอร์ดของแต่ละบทบาท
        \item \textbf{ผลลัพธ์ที่ได้:} ผู้ใช้งานสามารถเข้าสู่ระบบผ่าน CMU Account ได้อย่างถูกต้อง
        \item \textbf{สถานะ:} ทำงานถูกต้อง
      \end{enumerate}
    \subsubsection{การเข้าใช้งานครั้งแรกผ่าน Google Account}
      \begin{enumerate}
        \item \textbf{วัตถุประสงค์:} เพื่อทดสอบว่าผู้ใช้งานสามารถเข้าสู่ระบบผ่าน Google Account ได้อย่างถูกต้อง
        \item \textbf{ขั้นตอนการทดสอบ:}
          \begin{enumerate}
            \item เปิดหน้าเข้าสู่ระบบของแพลตฟอร์ม
            \item คลิกที่ปุ่ม "Sign-In"
            \item เลือกประเภทบัญชีเป็น "Google Account"
            \item ป้อนข้อมูล Google Account ที่ถูกต้อง
            \item คลิกที่ปุ่ม "Login"
            \item รอให้ระบบทําการตรวจสอบและเข้าสู่ระบบ
            \item กรอกรายละเอียดการใช้งานครั้งแรก ตามที่แพลตฟอร์มร้องขอ
            \item คลิกที่ปุ่ม "Sign Up" เพื่อสร้างบัญชีผู้ใช้งานใหม่
            \item ผู้ใช้งานถูกนําส่งไปยังหน้าแดชบอร์ดของแต่ละบทบาท
          \end{enumerate}
        \item \textbf{ผลลัพธ์ที่คาดหวัง:} ผู้ใช้งานสามารถเข้าสู่ระบบผ่าน Google Account ได้อย่างถูกต้อง
        และถ้าเป็นการเข้าสู่ระบบครั้งแรกจะถูกนําไปยังหน้ากรอกรายละเอียดการใช้งานครั้งแรก และหลังจากกรอกข้อมูลครบถ้วน
        จะถูกนําส่งไปยังหน้าแดชบอร์ดของแต่ละบทบาท
        \item \textbf{ผลลัพธ์ที่ได้:} ผู้ใช้งานสามารถเข้าสู่ระบบผ่าน Google Account ได้อย่างถูกต้อง
        \item \textbf{สถานะ:} ทำงานถูกต้อง
      \end{enumerate}
  \subsection{การทดสอบความถูกต้องของการจัดการกระบวนวิชา}
    \qquad ในการทดสอบความถูกต้องของระบบแพลตฟอร์มการตรวจใบงานแบบกระดาษ (Paper Grader) ได้มีการทดสอบฟังก์ชันดังนี้
    \subsubsection{การสร้างกระบวนวิชา}
      \begin{enumerate}
        \item \textbf{วัตถุประสงค์:} เพื่อทดสอบว่าผู้ใช้งานสามารถสร้างกระบวนวิชาใหม่ได้
        \item \textbf{ขั้นตอนการทดสอบ:}
          \begin{enumerate}
            \item เข้าสู่ระบบด้วยบัญชีผู้ใช้งานในบทบาทผู้สอน (Instructor)
            \item ไปที่หน้า "Course Overview"
            \item คลิกที่ปุ่ม "สร้างกระบวนวิชาใหม่ (Create New Course)"
            \item กรอกข้อมูลกระบวนวิชาใหม่ในฟอร์มที่กำหนด
            \item คลิกที่ปุ่ม "สร้างกระบวนวิชา (Create Course)"
          \end{enumerate}
        \item \textbf{ผลลัพธ์ที่คาดหวัง:} บัญชีผู้ใช้งานในบทบาทผู้สอน (Instructor) สามารถสร้างกระบวนวิชาใหม่
        และถูกสร้างขึ้นกระบวนวิชาใหม่ที่ถูกสร้างขึ้นแสดงในรายการกระบวนวิชาในหน้า "Course Overview" ได้อย่างถูกต้อง
        \item \textbf{ผลลัพธ์ที่ได้:} กระบวนวิชาใหม่ถูกสร้างขึ้นและแสดงในรายการกระบวนวิชา
        \item \textbf{สถานะ:} ทำงานถูกต้อง
      \end{enumerate}
    \subsubsection{การแสดงผลรายการกระบวนวิชา}
      \begin{enumerate}
        \item \textbf{วัตถุประสงค์:} เพื่อทดสอบว่าผู้ใช้งานสามารถดูรายการกระบวนวิชาที่ตนเองเป็นผู้สอนได้อย่างถูกต้อง
        \item \textbf{ขั้นตอนการทดสอบ:}
          \begin{enumerate}
            \item เข้าสู่ระบบด้วยบัญชีผู้ใช้งานในบทบาทผู้สอน (Instructor)
            \item ไปที่หน้า "Course Overview"
          \end{enumerate}
        \item \textbf{ผลลัพธ์ที่คาดหวัง:} บัญชีผู้ใช้งานในบทบาทผู้สอน (Instructor) สามารถดูรายการกระบวนวิชาที่ตนเองเป็นผู้สอนได้อย่างถูกต้อง
        \item \textbf{ผลลัพธ์ที่ได้:} รายการกระบวนวิชาที่ตนเองเป็นผู้สอนแสดงอย่างถูกต้อง
        \item \textbf{สถานะ:} ทำงานถูกต้อง
      \end{enumerate}
  \subsection{การทดสอบความถูกต้องของการจัดการรายชื่อภายในกระบวนวิชา}
    \qquad ในการทดสอบความถูกต้องของระบบแพลตฟอร์มการตรวจใบงานแบบกระดาษ (Paper Grader) ได้มีการทดสอบฟังก์ชันดังนี้
    \subsubsection{การเพิ่มรายชื่อผู้ใช้งานในกระบวนวิชาแบบรายบุคคล}
      \begin{enumerate}
        \item \textbf{วัตถุประสงค์:} เพื่อทดสอบว่าผู้ใช้งานสามารถเพิ่มรายชื่อผู้ใช้งานในกระบวนวิชาได้อย่างถูกต้อง
        \item \textbf{ขั้นตอนการทดสอบ:}
          \begin{enumerate}
            \item เข้าสู่ระบบด้วยบัญชีผู้ใช้งานในบทบาทผู้สอน (Instructor)
            \item ไปที่หน้า "Course Overview" และเลือกกระบวนวิชาที่ต้องการเพิ่มรายชื่อผู้ใช้งาน
            \item เลือกที่เมนู "Roster"
            \item คลิกที่ปุ่ม "เพิ่มรายชื่อผู้ใช้งาน (Add Members)"
            \item เลือกตัวเลือก "เพิ่มรายชื่อผู้ใช้งานแบบรายบุคคล (Add Single User)"
            \item ป้อนข้อมูลรายชื่อผู้ใช้งานที่ต้องการเพิ่ม
            \item คลิกที่ปุ่ม "เพิ่ม (Add)"
          \end{enumerate}
        \item \textbf{ผลลัพธ์ที่คาดหวัง:} รายชื่อผู้ใช้งานที่ถูกเพิ่มแสดงในรายการรายชื่อผู้ใช้งานในกระบวนวิชาได้อย่างถูกต้อง
        \item \textbf{ผลลัพธ์ที่ได้:} รายชื่อผู้ใช้งานที่ถูกเพิ่มแสดงในรายการรายชื่อผู้ใช้งานในกระบวนวิชาได้อย่างถูกต้อง
        \item \textbf{สถานะ:} ทำงานถูกต้อง
      \end{enumerate}
    \subsubsection{การเพิ่มรายชื่อผู้ใช้งานในกระบวนวิชาแบบนำเข้าจากไฟล์}
      \begin{enumerate}
        \item \textbf{วัตถุประสงค์:} เพื่อทดสอบว่าผู้ใช้งานสามารถเพิ่มรายชื่อผู้ใช้งานในกระบวนวิชาได้อย่างถูกต้อง
        \item \textbf{ขั้นตอนการทดสอบ:}
          \begin{enumerate}
            \item เข้าสู่ระบบด้วยบัญชีผู้ใช้งานในบทบาทผู้สอน (Instructor)
            \item ไปที่หน้า "Course Overview" และเลือกกระบวนวิชาที่ต้องการเพิ่มรายชื่อผู้ใช้งาน
            \item เลือกที่เมนู "Roster"
            \item คลิกที่ปุ่ม "เพิ่มรายชื่อผู้ใช้งาน (Add Members)"
            \item เลือกตัวเลือก "นำเข้ารายชื่อผู้ใช้งานจากไฟล์ (Multiple Users)"
            \item เลือกต้นแบบไฟล์ที่จะนำเข้ามาใช้
            \item อัพโหลดไฟล์รายชื่อผู้ใช้งานที่ต้องการเพิ่มตามต้นแบบ
            \item ทำการแมพคอลัมน์ข้อมูลให้ตรงกับฟิลด์ที่ระบบกำหนด
            \item คลิกที่ปุ่ม "นำเข้า (Import)"
          \end{enumerate}
        \item \textbf{ผลลัพธ์ที่คาดหวัง:} รายชื่อผู้ใช้งานที่ถูกเพิ่มแสดงในรายการรายชื่อผู้ใช้งานในกระบวนวิชาได้อย่างถูกต้อง
        \item \textbf{ผลลัพธ์ที่ได้:} รายชื่อผู้ใช้งานที่ถูกเพิ่มแสดงในรายการรายชื่อผู้ใช้งานในกระบวนวิชาได้อย่างถูกต้อง
        \item \textbf{สถานะ:} ทำงานถูกต้อง
      \end{enumerate}
  \subsection{การทดสอบความถูกต้องของการจัดการงานที่มอบหมาย}
    \qquad ในการทดสอบความถูกต้องของระบบแพลตฟอร์มการตรวจใบงานแบบกระดาษ (Paper Grader) ได้มีการทดสอบฟังก์ชันดังนี้
    \subsubsection{การสร้างงานที่มอบหมาย}
      \begin{enumerate}
        \item \textbf{วัตถุประสงค์:} เพื่อทดสอบว่าผู้ใช้งานสามารถสร้างงานที่มอบหมายใหม่ได้อย่างถูกต้อง
        \item \textbf{ขั้นตอนการทดสอบ:}
          \begin{enumerate}
            \item เข้าสู่ระบบด้วยบัญชีผู้ใช้งานในบทบาทผู้สอน (Instructor)
            \item ไปที่หน้า "Course Overview" และเลือกกระบวนวิชาที่ต้องการสร้างงานที่มอบหมาย
            \item คลิกที่ปุ่ม "สร้างงานที่มอบหมายใหม่ (Create New Assignment)"
            \item กรอกข้อมูลงานที่มอบหมายใหม่ในฟอร์มที่กำหนด
            \item แนบไฟล์ใบงานที่ต้องการให้ผู้เรียนดาวน์โหลด และไฟล์เพิ่มเติมอื่นๆ (ถ้ามี)
            \item คลิกที่ปุ่ม "สร้างงานที่มอบหมาย (Create Assignment)"
          \end{enumerate}
        \item \textbf{ผลลัพธ์ที่คาดหวัง:} งานที่มอบหมายใหม่ถูกสร้างขึ้นและแสดงในรายการงานที่มอบหมายในกระบวนวิชาได้อย่างถูกต้อง
        \item \textbf{ผลลัพธ์ที่ได้:} งานที่มอบหมายใหม่ถูกสร้างขึ้นและแสดงในรายการงานที่มอบหมายในกระบวนวิชาได้อย่างถูกต้อง
        \item \textbf{สถานะ:} ทำงานถูกต้อง
      \end{enumerate}
    \subsubsection{การการปรับเวลาการส่งงานหรือการแก้ไขงานที่มอบหมาย}
      \begin{enumerate}
        \item \textbf{วัตถุประสงค์:} เพื่อทดสอบว่าผู้ใช้งานสามารถปรับเวลาการส่งงานที่มอบหมาย และแก้ไขงานที่มอบหมายได้อย่างถูกต้อง
        \item \textbf{ขั้นตอนการทดสอบ:}
          \begin{enumerate}
            \item เข้าสู่ระบบด้วยบัญชีผู้ใช้งานในบทบาทผู้สอน (Instructor)
            \item ไปที่หน้า "Course Overview" และเลือกกระบวนวิชาที่ต้องการดูรายการงานที่มอบหมาย
            \item เลือกที่เมนู "Assignments"
            \item เลือกงานที่มอบหมายที่ต้องการปรับเวลาการส่ง
            \item คลิกที่ปุ่ม "แก้ไขงานที่มอบหมาย (Edit Assignment)"
            \item ปรับเวลาการส่งงานตามที่ต้องการในฟิลด์ที่กำหนด หรือแก้ไขข้อมูลอื่นๆ ของงานที่มอบหมายตามต้องการ
            \item คลิกที่ปุ่ม "บันทึกการเปลี่ยนแปลง (Save)"
          \end{enumerate}
        \item \textbf{ผลลัพธ์ที่คาดหวัง:} สามารถปรับแก้ไขงานที่มอบหมายได้อย่างถูกต้อง
        และงานที่มอบหมายที่ถูกแก้ไขแสดงในรายการงานที่มอบหมายในกระบวนวิชาได้ถูกต้อง
        \item \textbf{ผลลัพธ์ที่ได้:} งานที่มอบหมายที่ถูกแก้ไขแสดงในรายการงานที่มอบหมายในกระบวนวิชาได้อย่างถูกต้อง
        \item \textbf{สถานะ:} ทำงานถูกต้อง
      \end{enumerate}
    \subsubsection{การแสดงรายการงานที่มอบหมายภายในกระบวนวิชา}
      \begin{enumerate}
        \item \textbf{วัตถุประสงค์:} เพื่อทดสอบว่าผู้ใช้งานในบทบาทผู้สอน (Instructor)
        สามารถดูรายการงานที่สร้างหรือรายการงานที่มอบหมายในกระบวนวิชาได้อย่างถูกต้อง
        \item \textbf{ขั้นตอนการทดสอบ:}
          \begin{enumerate}
            \item เข้าสู่ระบบด้วยบัญชีผู้ใช้งานในบทบาทผู้สอน (Instructor)
            \item ไปที่หน้า "Course Overview" และเลือกกระบวนวิชาที่ต้องการดูรายการงานที่มอบหมาย
            \item เลือกที่เมนู "Assignments"
            \item ดูรายการงานที่สร้าง หรือรายการงานที่มอบหมายทั้งหมดในกระบวนวิชา
            \item กดไอคอนขยายเพื่อดูรายละเอียดของงานที่มอบหมายแต่ละงานในรูปแบบของลำดับตอน (Sections)
          \end{enumerate}
        \item \textbf{ผลลัพธ์ที่คาดหวัง:} รายการงานที่สร้างหรือรายการงานที่มอบหมายทั้งหมดในกระบวนวิชาแสดงอย่างถูกต้อง
        และสามารถดูรายละเอียดของงานที่มอบหมายแต่ละงานในรูปแบบของลำดับตอน (Sections) ได้อย่างถูกต้อง
        \item \textbf{ผลลัพธ์ที่ได้:} รายการงานที่มอบหมายในกระบวนวิชาแสดงอย่างถูกต้อง
        \item \textbf{สถานะ:} ทำงานถูกต้อง
      \end{enumerate}
    \subsubsection{การแสดงรายการงานที่มอบหมายภายในกระบวนวิชาที่อยู่ในระยะเวลาส่งงาน (Active Assignments)}
      \begin{enumerate}
        \item \textbf{วัตถุประสงค์:} เพื่อทดสอบว่า ผู้ใช้งานในบทบาทผู้สอน (Instructor) 
        สามารถดูรายการงานที่มอบหมายภายในกระบวนวิชาที่อยู่ในระยะเวลาส่งงานได้อย่างถูกต้อง
        \item \textbf{ขั้นตอนการทดสอบ:}
          \begin{enumerate}
            \item เข้าสู่ระบบด้วยบัญชีผู้ใช้งานในบทบาทผู้สอน (Instructor)
            \item ไปที่หน้า "Course Overview" และเลือกกระบวนวิชาที่ต้องการดูรายการงานที่มอบหมาย
            \item รายการงานที่มอบหมายในกระบวนวิชาที่อยู่ในระยะเวลาส่งงาน (Active Assignments)
            จะแสดงอย่างถูกต้อง
          \end{enumerate}
        \item \textbf{ผลลัพธ์ที่คาดหวัง:} รายการงานที่มอบหมายทั้งหมดในกระบวนวิชาที่อยู่ในระยะเวลาส่งงาน
        (Active Assignments) แสดงอย่างถูกต้อง
        \item \textbf{ผลลัพธ์ที่ได้:} รายการงานที่มอบหมายในกระบวนวิชา ที่อยู่ในระยะเวลาส่งงาน (Active Assignments)
        แสดงอย่างถูกต้อง
        \item \textbf{สถานะ:} ทำงานถูกต้อง
      \end{enumerate}
  \subsection{การทดสอบความถูกต้องของขั้นตอนการตรวจใบงาน}
    \qquad ในการทดสอบความถูกต้องของระบบแพลตฟอร์มการตรวจใบงานแบบกระดาษ (Paper Grader) ได้มีการทดสอบฟังก์ชันดังนี้
    \subsubsection{การกำหนดกรอบ (Edit Outline) ของใบงาน}
      \begin{enumerate}
        \item \textbf{วัตถุประสงค์:} เพื่อทดสอบว่าผู้ใช้งานในบทบาทผู้สอน (Instructor) สามารถกำหนดกรอบ (Outline)
        ของใบงานได้อย่างถูกต้อง
        \item \textbf{ขั้นตอนการทดสอบ:}
          \begin{enumerate}
            \item เข้าสู่ระบบด้วยบัญชีผู้ใช้งานในบทบาทผู้สอน (Instructor)
            \item ไปที่หน้า "Course Overview" และเลือกกระบวนวิชาที่ต้องการกำหนดกรอบของใบงาน
            \item เลือกที่เมนูรายการงาน "Assignments"
            \item เลือกงานที่มอบหมายที่ต้องการกำหนดกรอบของใบงาน
            \item คลิกที่ปุ่ม "กำหนดกรอบ (Edit Outline)"
            \item คลิกที่ปุ่ม "สร้างกรอบชื่อนักศึกษา (Create Student Name)" เพื่อสร้างกรอบชื่อนักศึกษา
            \item คลิกที่ปุ่ม "สร้างกรอบรหัสนักศึกษา (Create Student ID)" เพื่อสร้างกรอบรหัสนักศึกษา
            \item คลิกที่ปุ่ม "สร้างกรอบคำถาม (Create Question)" เพื่อสร้างกรอบคำถาม
            \item กรอกรายละเอียดของคำถามในฟิลด์ที่กำหนด
            \item คลิกที่ไอคอน "เครื่องหมายบวก" บนแถบด้านขวาหลังคำถามนั้นๆ เพื่อเพิ่มคำถามย่อย (Sub-question)
            \item กรอกรายละเอียดของคำถามย่อยในฟิลด์ที่กำหนด
            \item ลากและวางกรอบที่สร้างขึ้นไปยังตำแหน่ง
            \item คลิกที่ปุ่ม "บันทึกกรอบ (Save Outline)"
          \end{enumerate}
        \item \textbf{ผลลัพธ์ที่คาดหวัง:} กรอบของใบงานที่สร้างขึ้นในแต่ละประเภทถูกสร้างขึ้นถูกต้อง และแสดงรายละเอียดถูกต้อง
        \item \textbf{ผลลัพธ์ที่ได้:} กรอบของใบงานที่สร้างขึ้นในแต่ละประเภทถูกสร้างขึ้นถูกต้อง และแสดงรายละเอียดถูกต้อง
        \item \textbf{สถานะ:} ทำงานถูกต้อง
      \end{enumerate}
    \subsubsection{การอัพโหลดใบงานที่สแกนแล้วเพื่อเตรียมตรวจ (Manage Submissions)}
      \begin{enumerate}
        \item \textbf{วัตถุประสงค์:} เพื่อทดสอบว่าผู้ใช้งานในบทบาทผู้สอน (Instructor) สามารถอัพโหลดใบงานที่สแกนแล้ว
        เพื่อเตรียมตรวจได้อย่างถูกต้อง
        \item \textbf{ขั้นตอนการทดสอบ:}
          \begin{enumerate}
            \item เข้าสู่ระบบด้วยบัญชีผู้ใช้งานในบทบาทผู้สอน (Instructor)
            \item ไปที่หน้า "Course Overview" และเลือกกระบวนวิชาที่ต้องการอัพโหลดใบงานที่สแกนแล้ว
            \item เลือกที่เมนูรายการงาน "Assignments"
            \item เลือกงานที่มอบหมายที่ต้องการอัพโหลดใบงานที่สแกนแล้ว
            \item คลิกที่ปุ่ม "จัดการการส่งงาน (Manage Submissions)"
            \item คลิกที่ปุ่ม "อัพโหลดใบงาน (Upload Submissions) หรือลากจากไฟล์แล้ววางที่พื้นที่นี้"
            \item คลิกที่ปุ่ม "อัพโหลด (Upload)"
            \item รอให้ระบบทําการอัพโหลดใบงานที่อัพโหลด และประมวลผลใบงานโดยการแยกใบงานแต่ละใบออกจากกัน
            \item แสดงรายการใบงานที่อัพโหลดในหน้าจัดการการส่งงาน (Manage Submissions) ตามจำนวนที่แยกใบงานได้
          \end{enumerate}
        \item \textbf{ผลลัพธ์ที่คาดหวัง:} ใบงานถูกอัพโหลดและแสดงในรายการใบงานที่แยกตามจำนวนใบงานได้อย่างถูกต้อง
        \item \textbf{ผลลัพธ์ที่ได้:} ใบงานถูกอัพโหลดและแสดงในรายการใบงานที่แยกตามจำนวนใบงานได้อย่างถูกต้อง
        \item \textbf{สถานะ:} ทำงานถูกต้อง
      \end{enumerate}
    \subsubsection{การสร้างและแก้ไขเกณฑ์การให้คะแนน (Rubric)}
      \begin{enumerate}
        \item \textbf{วัตถุประสงค์:} เพื่อทดสอบว่าผู้ใช้งานในบทบาทผู้สอน (Instructor)
        สามารถสร้างและแก้ไขเกณฑ์การให้คะแนน (Rubric) ได้อย่างถูกต้อง
        \item \textbf{ขั้นตอนการทดสอบ:}
          \begin{enumerate}
            \item เข้าสู่ระบบด้วยบัญชีผู้ใช้งานในบทบาทผู้สอน (Instructor)
            \item ไปที่หน้า "Course Overview" และเลือกกระบวนวิชาที่ต้องการสร้างเกณฑ์การให้คะแนน
            \item เลือกที่เมนูรายการงาน "Assignments"
            \item เลือกที่เมนูการตรวจใบงาน "Grade Submissions"
            \item เลือกการตรวจแบบใบงานแบบรายบุคคล หรือการตรวจแบบรายข้อ
            \item หรือเลือกที่เมนูการกำหนดเอ้าท์ไลน์ "Edit outline"
            \item คลิกที่แถบเมนูด้านขวา "Rubric" เพื่อเข้าสู่หน้าสร้างเกณฑ์การให้คะแนน
            \item คลิกที่ปุ่ม "สร้างเกณฑ์การให้คะแนนใหม่ (Create New Rubric)"
            \item กรอกข้อมูลเกณฑ์การให้คะแนนใหม่ในฟอร์มที่กำหนด
            \item หรือเลือกเกณฑ์การให้คะแนนที่มีอยู่แล้วเพื่อทำการแก้ไข
            \item ทำการแก้ไขข้อมูลเกณฑ์การให้คะแนนตามต้องการ
            \item หรือสลับลำดับเกณฑ์การให้คะแนนโดยการลากและวาง
            \item หรือลบเกณฑ์การให้คะแนนโดยการคลิกที่ไอคอนรูปถังขยะ
          \end{enumerate}
        \item \textbf{ผลลัพธ์ที่คาดหวัง:} เกณฑ์การให้คะแนนใหม่ถูกสร้างขึ้นและแสดงในรายการเกณฑ์การให้คะแนนได้อย่างถูกต้อง
        หรือเกณฑ์การให้คะแนนที่ถูกแก้ไขแสดงในรายการเกณฑ์การให้คะแนนได้อย่างถูกต้อง
        \item \textbf{ผลลัพธ์ที่ได้:} เกณฑ์การให้คะแนนใหม่ถูกสร้างขึ้นและแสดงในรายการเกณฑ์การให้คะแนนได้อย่างถูกต้อง
        หรือเกณฑ์การให้คะแนนที่ถูกแก้ไขแสดงในรายการเกณฑ์การให้คะแนนได้อย่างถูกต้อง
      \end{enumerate}
  \subsection{การทดสอบความถูกต้องของการให้คะแนนใบงาน}
    \qquad ในการทดสอบความถูกต้องของระบบแพลตฟอร์มการตรวจใบงานแบบกระดาษ (Paper Grader) ได้มีการทดสอบฟังก์ชันดังนี้
    \subsubsection{การตรวจหรือให้เกณฑ์การให้คะแนน (Rubric) ในการตรวจใบงาน}
      \begin{enumerate}
        \item \textbf{วัตถุประสงค์:} เพื่อทดสอบว่าผู้ใช้งานสามารถให้เกณฑ์การให้คะแนน (Rubric) ที่ได้จากการสร้างเกณฑ์การให้คะแนน
        ในการตรวจใบงานได้อย่างถูกต้อง
        \item \textbf{ขั้นตอนการทดสอบ:}
          \begin{enumerate}
            \item เข้าสู่ระบบด้วยบัญชีผู้ใช้งานในบทบาทผู้สอน (Instructor)
            \item ไปที่หน้า "Course Overview" และเลือกกระบวนวิชาที่ต้องการตรวจใบงาน
            \item เลือกที่เมนูรายการงาน "Assignments"
            \item เลือกที่เมนูการตรวจใบงาน "Grade Submissions"
            \item เลือกการตรวจแบบใบงานแบบรายบุคคล หรือการตรวจแบบรายข้อ
            \item ใช้เกณฑ์การให้คะแนน (Rubric) ที่ถูกสร้างขึ้นในการให้คะแนนใบงาน
            โดยเลือกเกณฑ์การให้คะแนนจากแถบเมนูด้านขวา "Rubric"
            \item ให้คะแนนตามเกณฑ์การให้คะแนนที่เลือกไว้โดยการคลิกเลือกระดับคะแนนในแต่ละเกณฑ์
            \item หรือให้คะแนนตามเกณฑ์การให้คะแนนที่เลือกไว้โดยการใช้คีย์บอร์ด (Shortcut key) เลข 1-9 เพื่อเลือกเกณฑ์การให้คะแนนในแต่ละข้อ
            \item กดคำถามที่ถัดไปเพื่อตรวจใบงานถัดไป โดยที่กดปุ่มใบงานถัดไปโดยตรงหรือใช้คีย์บอร์ด (Shortcut key)
            ลูกศรขึ้น-ลง เพื่อตรวจคำถามถัดไปในใบงานเดียวกัน และกดปุ่มลูกศรซ้าย-ขวา เพื่อตรวจใบงานถัดไป
          \end{enumerate}
        \item \textbf{ผลลัพธ์ที่คาดหวัง:} ผู้ใช้งานสามารถใช้เกณฑ์การให้คะแนน (Rubric)
        ในการตรวจใบงานได้อย่างถูกต้อง และผลการตรวจใบงานแสดงอย่างถูกต้องตามเกณฑ์การให้คะแนนที่ใช้
        \item \textbf{ผลลัพธ์ที่ได้:} ผู้ใช้งานสามารถใช้เกณฑ์การให้คะแนน (Rubric)
        ในการตรวจใบงานได้อย่างถูกต้อง และผลการตรวจใบงานแสดงอย่างถูกต้องตามเกณฑ์การให้คะแนนที่ใช้
        \item \textbf{สถานะ:} ทำงานถูกต้อง
      \end{enumerate}
    \subsubsection{การสร้างเกณฑ์การให้คะแนน (Rubric) หลังการตรวจใบงาน}
      \begin{enumerate}
        \item \textbf{วัตถุประสงค์:} เพื่อทดสอบว่าผู้ใช้งานสามารถสร้างเกณฑ์การให้คะแนน (Rubric) ได้อย่างถูกต้อง
        และเกณฑ์การให้คะแนนใหม่ที่ถูกสร้างขึ้นได้ถูกปรับในการให้คะแนนใบงานที่ได้รับการตรวจไปแล้ว
        \item \textbf{ขั้นตอนการทดสอบ:}
          \begin{enumerate}
            \item เข้าสู่ระบบด้วยบัญชีผู้ใช้งานในบทบาทผู้สอน (Instructor)
            \item ไปที่หน้า "Course Overview" และเลือกกระบวนวิชาที่ต้องการสร้างเกณฑ์การให้คะแนน
            \item เลือกที่เมนูรายการงาน "Assignments"
            \item เลือกที่เมนูการตรวจใบงาน "Grade Submissions"
            \item เลือกการตรวจแบบใบงานแบบรายบุคคล หรือการตรวจแบบรายข้อ
            \item หรือเลือกที่เมนูการกำหนดเอ้าท์ไลน์ "Edit outline"
            \item คลิกที่แถบเมนูด้านขวา "Rubric" เพื่อเข้าสู่หน้าสร้างเกณฑ์การให้คะแนน
            \item คลิกที่ปุ่ม "สร้างเกณฑ์การให้คะแนนใหม่ (Create New Rubric)"
            \item กรอกข้อมูลเกณฑ์การให้คะแนนใหม่ในฟอร์มที่กำหนด
          \end{enumerate}
        \item \textbf{ผลลัพธ์ที่คาดหวัง:} เกณฑ์การให้คะแนนใหม่ที่ถูกสร้างขึ้นหลังการตรวจใบงานแสดงในรายการเกณฑ์การให้คะแนนได้อย่างถูกต้อง
        และเกณฑ์การให้คะแนนใหม่ที่ถูกสร้างขึ้นถูกปรับใช้ในการให้คะแนนใบงานที่ได้รับการตรวจไปแล้วอย่างถูกต้อง
        \item \textbf{ผลลัพธ์ที่ได้:} เกณฑ์การให้คะแนนใหม่ถูกสร้างขึ้นและถูกปรับใช้ในการให้คะแนนใบงานที่ได้รับการตรวจไปแล้วอย่างถูกต้อง
        \item \textbf{สถานะ:} ทำงานถูกต้อง
      \end{enumerate}
  \subsection{การทดสอบความถูกต้องของการแสดงผลสถิติ}
    \qquad ในการทดสอบความถูกต้องของระบบแพลตฟอร์มการตรวจใบงานแบบกระดาษ (Paper Grader) ได้มีการทดสอบฟังก์ชันดังนี้
    \subsubsection{การแสดงผลสถิติของแต่ละงานที่มอบหมายภายในกระบวนวิชา (Assignment Statistics)}
      \begin{enumerate}
        \item \textbf{วัตถุประสงค์:} เพื่อทดสอบว่าผู้ใช้งานในบทบาทผู้สอน (Instructor)
        สามารถดูสถิติของแต่ละงานที่มอบหมายภายในกระบวนวิชาได้อย่างถูกต้อง
        \item \textbf{ขั้นตอนการทดสอบ:}
          \begin{enumerate}
            \item เข้าสู่ระบบด้วยบัญชีผู้ใช้งานในบทบาทผู้สอน (Instructor)
            \item ไปที่หน้า "Course Overview" และเลือกกระบวนวิชาที่ต้องการดูรายการงานที่มอบหมาย
            \item เลือกที่เมนู "Statistics"
            \item หรือเลือกที่เมนู "Assignments" แล้วกดที่งานที่มอบหมายที่ต้องการดูสถิติ (Assignment)
            \item เลือกรายการงานที่ต้องการดูสถิติ และเลือกลำดับตอน (Section) ที่ต้องการดูสถิติ
            \item แสดงกราฟแท่งแสดงสถิติคะแนน (Histogram) ของงานที่มอบหมาย
            \item แสดงค่าสถิติต่างๆ ได้แก่ ค่าต่ำสุด (Min), ค่าสูงสุด (Max), ค่าเฉลี่ย (Mean), ค่ามัธยฐาน (Median),
            ส่วนเบี่ยงเบนมาตรฐาน (Standard Deviation) และจำนวนผู้ส่งงาน (Total Submissions)
            \item แสดงรายการคำถามแต่ละข้อในงานที่มอบหมาย พร้อมด้วยสถิติคะแนนของแต่ละข้อ
          \end{enumerate}
        \item \textbf{ผลลัพธ์ที่คาดหวัง:} สามารถแสดงสถิติของแต่ละงานที่มอบหมายภายในกระบวนวิชาของแต่ละลำดับตอน
        (Section) ได้อย่างถูกต้อง
        \item \textbf{ผลลัพธ์ที่ได้:} แสดงสถิติของแต่ละงาน ที่มอบหมายภายในกระบวนวิชาของแต่ละลำดับตอน
        (Section) ได้อย่างถูกต้อง
        \item \textbf{สถานะ:} ทำงานถูกต้อง
      \end{enumerate}
    \subsubsection{การแสดงผลสถิติของงานที่มอบหมาย (Review Grade)}
      \begin{enumerate}
        \item \textbf{วัตถุประสงค์:} เพื่อทดสอบว่าผู้ใช้งานในบทบาทผู้สอน (Instructor)
        สามารถดูสถิติของงานที่มอบหมายได้อย่างถูกต้อง
        \item \textbf{ขั้นตอนการทดสอบ:}
          \begin{enumerate}
            \item เข้าสู่ระบบด้วยบัญชีผู้ใช้งานในบทบาทผู้สอน (Instructor)
            \item ไปที่หน้า "Course Overview" และเลือกกระบวนวิชาที่ต้องการดูรายการงานที่มอบหมาย
            \item เลือกที่เมนู "Assignments"
            \item เลือกงานที่มอบหมายที่ต้องการดูสถิติ (Review Grade)
            \item ปรับจำนวนของกราฟแท่ง (Bin Size) แสดงสถิติคะแนน (Histogram) ตามต้องการไม่เกินที่กำหนด
            \item แสดงกราฟแท่งแสดงสถิติคะแนน (Histogram) ของงานที่มอบหมาย
            \item แสดงค่าสถิติต่างๆ ได้แก่ ค่าต่ำสุด (Min), ค่าสูงสุด (Max), ค่าเฉลี่ย (Mean), ค่ามัธยฐาน (Median),
            ส่วนเบี่ยงเบนมาตรฐาน (Standard Deviation) และจำนวนผู้ส่งงาน (Total Submissions)
            \item แสดงรายการนักศึกษาทั้งหมดภายในคอร์ส ที่ได้รับมอบหมายงานนี้พร้อมด้วยรายละเอียดต่างๆ
            ได้แก่ ชื่อ-นามสกุล (Name), เมลล์ (Email), ลำดับตอน (Section), คะแนน (Score),
            สถานะการตรวจ (Graded), สถานะการส่งงาน (Submitted) และวันที่ส่งงาน (Time)
          \end{enumerate}
        \item \textbf{ผลลัพธ์ที่คาดหวัง:} สามารถแสดงสถิติของงานที่มอบหมายตามจำนวนการปรับจำนวนของกราฟแท่ง
        (Bin Size) เผื่อแสดงสถิติคะแนน (Histogram) ได้อย่างถูกต้อง และรายละเอียดต่างๆ
        ของนักศึกษาที่ได้รับมอบหมายงานนี้แสดงได้อย่างถูกต้อง
        \item \textbf{ผลลัพธ์ที่ได้:} แสดงสถิติของงานที่มอบหมายและรายละเอียดต่างๆ ได้อย่างถูกต้อง
        \item \textbf{สถานะ:} ทำงานถูกต้อง
      \end{enumerate}
  \subsection{การทดสอบความถูกต้องของการนำออกคะแนน}
    \qquad ในการทดสอบความถูกต้องของระบบแพลตฟอร์มการตรวจใบงานแบบกระดาษ (Paper Grader) ได้มีการทดสอบฟังก์ชันดังนี้
    \subsubsection{การนำออกคะแนนในรูปแบบแต่ละรายการงาน}
      \begin{enumerate}
        \item \textbf{วัตถุประสงค์:} เพื่อทดสอบความถูกต้องว่าผู้ใช้งานในบทบาทผู้สอน (Instructor)
        สามารถนำออกคะแนนในรูปแบบแต่ละรายการงานภายในคอร์สๆ นั้น ในรูปแบบไฟล์สกุล (.Excel) ได้อย่างถูกต้อง
        \item \textbf{ขั้นตอนการทดสอบ:}
          \begin{enumerate}
            \item เข้าสู่ระบบด้วยบัญชีผู้ใช้งานในบทบาทผู้สอน (Instructor)
            \item ไปที่หน้า "Course Overview" และเลือกกระบวนวิชาที่ต้องการดูรายการงานที่มอบหมาย
            \item เลือกที่เมนู "นำออกคะแนน (Data Exports)"
            \item เลือกรายการงานที่ต้องการนำออกคะแนน ในรูปแบบแต่ละรายการงาน (Per Assignment)
            \item คลิกที่ปุ่ม "นำออกคะแนน (Export Grades)"
            \item รอให้ระบบทําการสร้างไฟล์นำออกคะแนน
            \item รายการประวัติการนำออกคะแนนจะแสดงในตารางประวัติการนำออกคะแนน
            \item คลิกที่ไอคอนปุ่มดาวน์โหลด (Download) เพื่อดาวน์โหลดไฟล์นำออกคะแนน
            \item เปิดไฟล์นำออกคะแนนที่ดาวน์โหลดมา เพื่อตรวจสอบความถูกต้องของข้อมูลคะแนน
          \end{enumerate}
        \item \textbf{ผลลัพธ์ที่คาดหวัง:} สามารถนำออกคะแนนในรูปแบบแต่ละรายการงานภายในคอร์สๆ นั้น
        และรายการประวัติการนำออกคะแนนแสดงได้อย่างถูกต้อง
        \item \textbf{ผลลัพธ์ที่ได้:} สามารถนำออกคะแนนในรูปแบบแต่ละรายการงานภายในคอร์สๆ นั้น ได้อย่างถูกต้อง
        รายการประวัติการนำออกคะแนนแสดงได้ถูกต้อง
        \item \textbf{สถานะ:} ทำงานถูกต้อง
      \end{enumerate}
  \subsection{การทดสอบความถูกต้องของการแสดงกระบวนวิชาและงานที่มอบหมายสำหรับนักศึกษา}
    \qquad ในการทดสอบความถูกต้องของระบบแพลตฟอร์มการตรวจใบงานแบบกระดาษ (Paper Grader) ได้มีการทดสอบฟังก์ชันดังนี้
    \subsubsection{การแสดงรายการคอร์สที่ลงทะเบียน}
      \begin{enumerate}
        \item \textbf{วัตถุประสงค์:} เพื่อทดสอบว่าผู้ใช้งานในบทบาทนักศึกษา (Student)
        สามารถดูรายการคอร์สที่ตนเองลงทะเบียนได้อย่างถูกต้อง
        \item \textbf{ขั้นตอนการทดสอบ:}
          \begin{enumerate}
            \item เข้าสู่ระบบด้วยบัญชีผู้ใช้งานในบทบาทนักศึกษา (Student)
            \item ไปที่หน้า "Courses"
            \item รายการคอร์สที่ตนเองลงทะเบียนทั้งหมดจะแสดงอย่างถูกต้อง
          \end{enumerate}
        \item \textbf{ผลลัพธ์ที่คาดหวัง:} รายการคอร์สที่ตนเองลงทะเบียนแสดงอย่างถูกต้อง
        \item \textbf{ผลลัพธ์ที่ได้:} รายการคอร์สที่ตนเองลงทะเบียนแสดงอย่างถูกต้อง
        \item \textbf{สถานะ:} ทำงานถูกต้อง
      \end{enumerate}
    \subsubsection{การแสดงรายการงานที่อยู่ในกำหนดส่งงานและเลยกำหนดการส่งงาน}
      \begin{enumerate}
        \item \textbf{วัตถุประสงค์:} เพื่อทดสอบว่าผู้ใช้งานในบทบาทนักศึกษา (Student)
        สามารถดูรายการงานที่อยู่ในกำหนดส่งงานและเลยกำหนดการส่งงานได้อย่างถูกต้อง
        \item \textbf{ขั้นตอนการทดสอบ:}
          \begin{enumerate}
            \item เข้าสู่ระบบด้วยบัญชีผู้ใช้งานในบทบาทนักศึกษา (Student)
            \item ไปที่หน้า "Dashboard"
            \item รายการงานที่อยู่ในกำหนดส่งงานและเลยกำหนดการส่งงานจะแสดงอย่างถูกต้อง
            \item กดไอคอนขยายเพื่อดูรายละเอียดของงานที่มอบหมายแต่ละงาน
            \item ตรวจสอบข้อมูลรายละเอียดงานที่แสดงในหน้า Modal
            \item ดาวน์โหลดไฟล์ใบงานต้นฉบับที่แนบมา และไฟล์เพิ่มเติมอื่นๆ (ถ้ามี)
            \item ดูใบงานที่สแกนเรียบร้อยแล้วที่ถูกอับโหลดขึ้นระบบ (ถ้ามี)
            \item ปิดหน้า Modal
          \end{enumerate}
        \item \textbf{ผลลัพธ์ที่คาดหวัง:} รายการงานที่อยู่ในกำหนดส่งงานและเลยกำหนดการส่งงานแสดงอย่างถูกต้อง
        \item \textbf{ผลลัพธ์ที่ได้:} รายการงานที่อยู่ในกำหนดส่งงานและเลยกำหนดการส่งงานแสดงอย่างถูกต้อง
        \item \textbf{สถานะ:} ทำงานถูกต้อง
      \end{enumerate}
  \subsection{การทดสอบความถูกต้องของการส่งใบงานของนักศึกษา}
    \qquad ในการทดสอบความถูกต้องของระบบแพลตฟอร์มการตรวจใบงานแบบกระดาษ (Paper Grader) ได้มีการทดสอบฟังก์ชันดังนี้
    \subsubsection{การส่งงานที่อยู่ในกำหนดการส่ง}
      \begin{enumerate}
        \item \textbf{วัตถุประสงค์:} เพื่อทดสอบว่าผู้ใช้งานในบทบาทนักศึกษา (Student)
        สามารถส่งงานที่อยู่ในกำหนดการส่งได้อย่างถูกต้อง
        \item \textbf{ขั้นตอนการทดสอบ:}
          \begin{enumerate}
            \item เข้าสู่ระบบด้วยบัญชีผู้ใช้งานในบทบาทนักศึกษา (Student)
            \item ไปที่หน้า "Dashboard"
            \item เลือกรายการงานที่อยู่ในกำหนดการส่งงานที่ต้องการส่ง
            \item คลิกที่ไอคอนอับโหลด
            \item Modal สำหรับการส่งงานจะแสดงขึ้นมา
            \item อัพโหลดไฟล์ใบงานที่สแกนเรียบร้อยแล้วตามต้นแบบที่ระบบกำหนด สกุล (.PDF)
            \item ตรวจสอบข้อมูลรายละเอียดงานที่จะแสดงในหน้า Modal
            \item คลิกที่ปุ่ม "ยืนยันการส่งงาน (Confirm Submission)"
            \item ระแบบแสดงการแจ้งเตือนการส่งงานสำเร็จ
          \end{enumerate}
        \item \textbf{ผลลัพธ์ที่คาดหวัง:} ผู้ใช้งานในบทบาทนักศึกษา (Student) สามารถส่งงานที่อยู่ในกำหนดการส่งได้อย่างถูกต้อง
        และแสดงเเจ้งเตือนการส่งงานสำเร็จ (Notification) อย่างถูกต้อง
        \item \textbf{ผลลัพธ์ที่ได้:} งานที่อยู่ในกำหนดการส่ง ถูกส่งแบบถูกต้อง และแสดงการแจ้งเตือนการส่งงานถูกต้อง
        \item \textbf{สถานะ:} ทำงานถูกต้อง
      \end{enumerate}