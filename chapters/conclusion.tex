\chapter{\ifenglish Conclusions and Discussions\else บทสรุปและข้อเสนอแนะ\fi}

\section{\ifenglish Conclusions\else สรุปผล\fi}
  \qquad ในบทนี้จะสรุปถึงข้อจํากัดของแพลตฟอร์มสำหรับการตรวจใบงานแบบกระดาษ (PaperGrader)
  ในด้านต่างๆ ที่ระบบมีในเนื้อหาส่วนนี้ รวมถึงข้อเสนอแนะและแนวทางการพัฒนาต่อในอนาคต

\section{\ifenglish Challenges\else ปัญหาที่พบและแนวทางการแก้ไข\fi}
  \qquad จากการพัฒนาแพลตฟอร์มสำหรับการตรวจใบงานแบบกระดาษ (PaperGrader) ทางผู้พัฒนาพบเจอปัญหาสําคัญ
  ที่ส่งผลต่อประสิทธิภาพการทำงานของแพลตฟอร์มและความคล่องตัวของโครงการดังนี้
  \subsection{ปัญหาที่พบ}
    \begin{enumerate}
      \item ขาดความเข้าใจเชิงลึกเกี่ยวกับระบบ ทีมพัฒนามีข้อมูลเกี่ยวกับระบบไม่มากพอ ส่งผลให้การออกแบบ
      และพัฒนาต้องมีการเปลี่ยนแปลงบ่อยครั้ง ซึ่งกระทบต่อประสิทธิภาพของกระบวนการพัฒนา
      \item ขาดการออกแบบ UX/UI ที่เหมาะสม ทีมพัฒนาไม่ได้ออกแบบ UX/UI ให้ครอบคลุมมากพอ ส่งผลให้ระหว่างการพัฒนา
      ต้องมีการเปลี่ยนแปลงบ่อยครั้ง ซึ่งทําให้การพัฒนาล่าช้า
      \item ความไม่แน่นอนของข้อกําหนด ข้อกําหนดของระบบมีการเปลี่ยนแปลงบ่อยครั้ง ส่งผลให้การออกแบบระบบต้อง
      มีการปรับเปลี่ยนตาม ทําให้การพัฒนาล่าช้า
    \end{enumerate}
  \subsection{แนวทางการแก้ไข}
    \begin{enumerate}
      \item เพื่อแก้ไขปัญหาเหล่านี้ ควรมีการกําหนดข้อกําหนดให้แน่นอนตั้งแต่เริ่มต้น โดยใช้แนวทาง Agile ในการพัฒนา
      เพื่อลดผลกระทบจากการเปลี่ยนแปลง
      \item ควรมีการออกแบบ UX/UI ใบแบบที่ครอบคลุมและเหมาะสมตั้งแต่เริ่มต้น เพื่อลดการเปลี่ยนแปลงในระหว่างการพัฒนา
      \item ควรมีการสื่อสารที่ชัดเจนระหว่างทีมพัฒนาและผู้มีส่วนได้ส่วนเสีย เพื่อให้เข้าใจข้อกําหนด
      และความต้องการของระบบอย่างถูกต้อง
    \end{enumerate}

\section{\ifenglish Suggestions and further improvements\else ข้อเสนอแนะและแนวทางการพัฒนาต่อ\fi}
  \qquad เพื่อให้แพลตฟอร์มสำหรับการตรวจใบงานแบบกระดาษ (PaperGrader) ที่มีประสิทธิภาพมากขึ้นในอนาคต
  ควรดําเนินการตามแนวทางดังต่อไปนี้
  \begin{enumerate}
    \item ปรับปรุงประสิทธิภาพการทำงานของระบบ Optical Character Recognition (OCR) ในการจับคู่ใบงานกับรายชื่อนักศึกษา
    ให้มีความแม่นยำและรวดเร็วยิ่งขึ้น เพราะในปัจจุบันมีการใช้ tesseract ซึ่งผลลัพธ์ในการจับคู่นั้นมีความแม่นยำค่อนข้างต่ำ
    \item พัฒนาให้การแสดงผลบนอุปกรณ์พกพารองรับได้อย่างสมบูรณ์ เพื่อความสะดวกสบายในการใช้งานทุกแพลตฟอร์ม
    \item เพิ่มฟังก์ชันการตรวจใบงานในรูปแบบที่เป็นกลุ่ม (Group Grading) สำหรับคำตอบที่เหมือนกัน เพื่อช่วยลดระยะเวลา
    ในการตรวจ
    \item เพิ่มบทบาทผู้ช่วยผู้อาจารย์ (TA) โดยกำหนดบทบาท เพิ่มรายชื่อนักศึกษาได้ สามารถช่วยตรวจได้ โดยการช่วยตรวจนั้นสามารถกำหนดได้ว่าจะให้ผู้ช่วยผู้อาจารย์คนนั้นๆ สามารถตรวจงานที่ได้หมอบหมายอันไหนได้บ้าง
    \item เพิ่มระบบการร้องขอตรวจใหม่ (Regrade request) เป็นระบบที่ให้นักศึกษาสามารถส่งคำร้องขอได้อาทิ เช่น เมื่อเราตรวจสอบผ่านเกณฑ์การให้คะแนนและกับเพื่อนๆ แล้วคะแนนส่วนนี้อาจารย์มีการตรวจผิด ให้สามารถกดเลือกคำถาม พร้อมบอกเหตุผล เพื่อร้องขอการตรวจใหม่ได้  
    \item เพิ่มระบบการจัดเก็บข้อมูล archive โดยการ Export ไฟล์ Data json ไฟล์แรกเป็น จะเป็นข้อมูลของรายชื่อภายในคอร์ส (Roster) และ ไฟล์ที่สองจะเป็นไฟล์ข้อมูลของงานที่ได้หมอบหมายพร้อมข้อมูลเกณฑ์การให้คะแนน
  \end{enumerate}