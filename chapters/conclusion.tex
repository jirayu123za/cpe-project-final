\chapter{\ifenglish Conclusions and Discussions\else บทสรุปและข้อเสนอแนะ\fi}

\section{\ifenglish Conclusions\else สรุปผล\fi}

นศ. ควรสรุปถึงข้อจำกัดของระบบในด้านต่างๆ ที่ระบบมีในเนื้อหาส่วนนี้ด้วย

\section{\ifenglish Challenges\else ปัญหาที่พบและแนวทางการแก้ไข\fi}

จากการพัฒนาระบบ PaperGrader พบปัญหาสําคัญที่ส่งผลต่อประสิทธิภาพและความคล่องตัวของโครงการ
\begin{itemize}
  \item ขาดความเข้าใจเชิงลึกเกี่ยวกับระบบ ทีมพัฒนามีข้อมูลเกี่ยวกับระบบไม่มากพอ ส่งผลให้การออกแบบและพัฒนาต้องมีการเปลี่ยนแปลงบ่อยครั้ง ซึ่งกระทบต่อประสิทธิภาพของกระบวนการพัฒนา
  \item ขาดการออกแบบ UX/UI ที่เหมาะสม ทีมพัฒนาไม่ได้ออกแบบ UX/UI ให้ครอบคลุมมากพอ ส่งผลให้ระหว่างการพัฒนาต้องมีการเปลี่ยนแปลงบ่อยครั้ง ซึ่งทําให้การพัฒนาล่าช้า
  \item ความไม่แน่นอนของข้อกําหนด ข้อกําหนดของระบบมีการเปลี่ยนแปลงบ่อยครั้ง ส่งผลให้การออกแบบระบบต้องมีการปรับเปลี่ยนตาม ทําให้การพัฒนาล่าช้า
\end{itemize}
เพื่อแก้ไขปัญหาเหล่านี้ ควรมีการกําหนดข้อกําหนดให้แน่นอนตั้งแต่เริ่มต้น ใช้แนวทาง Agile ในการพัฒนา
เพื่อลดผลกระทบจากการเปลี่ยนแปลง

\section{\ifenglish%
Suggestions and further improvements
\else%
ข้อเสนอแนะและแนวทางการพัฒนาต่อ
\fi
}

เพื่อให้ระบบ PaperGrader มีประสิทธิภาพมากขึ้นในอนาคต ควรดําเนินการตามแนวทางดังต่อไปนี้:
\begin{itemize}
  \item ปรับปรุงประสิทธิภาพของระบบ OCR ในการจับคู่ใบงานกับรายชื่อนักศึกษาให้มีความแม่นยำและรวดเร็วยิ่งขึ้น
  \item พัฒนาให้การแสดงผลบนอุปกรณ์พกพารองรับได้อย่างสมบูรณ์ เพื่อความสะดวกในการใช้งานทุกแพลตฟอร์ม
  \item เพิ่มฟังก์ชันการตรวจงานแบบกลุ่ม (Group Grading) สำหรับคำตอบที่เหมือนกัน เพื่อช่วยลดระยะเวลาในการตรวจ
\end{itemize}